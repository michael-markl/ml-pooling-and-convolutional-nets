\documentclass{article}
\usepackage[utf8]{inputenc}
\usepackage[english, ngerman]{babel}
\usepackage[a5paper,margin=1mm]{geometry}
\bibliographystyle{alphadin}

\newtheorem{theorem}{Theorem}
\newtheorem{definition}[theorem]{Definition}


\title{Neuronale Faltungsnetze und Pooling}
\author{Michael Markl}

\begin{document}
\maketitle


\section{Einführung}

Neuronale Netze werden bereits heutzutage auf zahlreiche Problemstellungen angewandt.
Dabei ist Computer Vision -- also die Verarbeitung und Analyse von Bildern -- eine der meist genutzten Technologien, die dadurch ermöglicht wird.
Ein klassisches Problem der Computer Vision ist, ein Bild zu einer bestimmten Kategorie zuzuordnen.
Bei der Erkennung von handschriftlichen Ziffern hat LeCun in~\cite{LeCun1989} aufgezeigt, dass das Reduzieren eines Netzwerks durch das Teilen von Parametern mehrere Neuronen zwischen zwei Layern die topologische 2D-Struktur von Bildern ausnutzen kann und dadurch schneller eine höhere 

% TODO:
Diese Arbeit gibt einen Einblick in Theorie und Praxis von Faltungsnetzen sowie die oft dabei verwendeten Pooling-Layer.
Als Grundlagen werden die Werke von Ian Goodfellow et al. in~\cite[Kapitel~9]{Goodfellow-et-al-2016} sowie von Ovidiu Calin in~\cite[Kapitel~15,16]{Calin2020} herangezogen.

\section{Faltung in Neuronalen Netzen}

Neuronale Netze.

\section{Faltungsnetze}





\begin{definition}[The Convolution Operation]
    
\end{definition}

\clearpage          % neue Seite für Literaturverzeichnis
\nocite*
\thispagestyle{empty}
\bibliography{literature}

\end{document}