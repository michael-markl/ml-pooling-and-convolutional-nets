\documentclass{article}
%\usepackage[a5paper,margin=5mm]{geometry}

\usepackage[utf8]{inputenc}
\usepackage[english, ngerman]{babel}
\usepackage{mathtools}
\usepackage{amsmath}
\usepackage{amssymb}
\usepackage{amsthm}
\usepackage{subcaption}


\usepackage{mathbbol}
\usepackage{color}

\usepackage{tikz}
\usetikzlibrary{matrix}

\DeclareSymbolFontAlphabet{\amsmathbb}{AMSb}%


\bibliographystyle{alphadin}

\newtheorem{theorem}{Theorem}[section]
\newtheorem{proposition}[theorem]{Proposition}
\theoremstyle{definition}
\newtheorem{definition}[theorem]{Definition}
\newcommand{\R}{\mathbb{R}}
\newcommand{\Z}{\mathbb{Z}}
\newcommand{\N}{\mathbb{N}}
\newcommand{\diff}{\,\textrm{d}}
\newcommand{\todo}[1]{{\color{red} #1}}

\newcommand{\norm}[1]{\left\lVert#1\right\rVert}
\newcommand{\abs}[1]{\left\lvert#1\right\rvert}
\newcommand{\avg}{\textnormal{avg}}



\title{Neuronale Faltungsnetze und Pooling}
\author{Michael Markl}

\begin{document}
\maketitle


\section{Einführung}

Neuronale Netze werden bereits heutzutage auf zahlreiche Problemstellungen angewandt.
Dabei ist Computer Vision -- also die Verarbeitung und Analyse von Bildern -- eine der meist genutzten Technologien, die dadurch ermöglicht wird.
Ein klassisches Problem der Computer Vision ist, ein Bild zu einer bestimmten Kategorie zuzuordnen.
Bei der Erkennung von handschriftlichen Ziffern hat LeCun in~\cite{LeCun1989} aufgezeigt, dass das Reduzieren eines Netzwerks durch das Teilen von Parametern mehrere Neuronen zwischen zwei Layern die topologische 2D-Struktur von Bildern ausnutzen kann und dadurch die Klassifizierung erheblich verbessern kann.

% TODO:
Diese Arbeit gibt einen Einblick in Theorie und Praxis von Faltungsnetzen sowie die oft dabei verwendeten Pooling-Layer.
Als Grundlagen werden die Werke von Ian Goodfellow et al. in~\cite[Kapitel~9]{Goodfellow-et-al-2016} sowie von Ovidiu Calin in~\cite[Kapitel~15,16]{Calin2020} herangezogen.

\section{Faltungslayer}

\subsection{Faltung und Kreuzkorrelation}

Bevor wir Faltungsnetze einführen, möchten wir zunächst die zugrundeliegenden mathematischen Konzepte einführen:
Faltung und Kreuzkorrelation.
Diese werden oft in der digitalen Signalverarbeitung genutzt, wo man zwischen kontinuierlichen und diskreten Signalen unterscheidet.

\newcommand{\llambda}{\lambda}
\newcommand{\cmeasure}{\mu_c}

\begin{definition}[Signale]
    Ein \emph{kontinuierliches Signal} ist eine Funktion $f: \R^d\rightarrow \R$.
    Im Kontext von kontinuierlichen Signalen wird das Lebesgue-Maß $\llambda$ auf $\Omega\coloneqq \R^d$ verwendet.

    Ein \emph{diskretes Signal} ist eine Funktion $f: \Z^d \rightarrow \R$.
    Bei diskreten Signalen wird das Zählmaß $\cmeasure$ auf $\Omega\coloneqq \Z^d$ verwendet.

    Ein Signal $f: \Omega \rightarrow \R$
    \begin{itemize}
        \item \emph{hat einen kompakten Träger}, falls es ein $R > 0$ gibt, sodass $f(x) = 0$ für $\norm{x}_\infty \geq R$ gilt,
        \item heißt \emph{$L^1$-endlich}, falls $\norm{f}_1 \coloneqq \int_{\Omega} \abs{f(x)} \diff \mu(x) < \infty$,
        \item heißt \emph{Energie-Signal}, falls $\norm{f}_2 \coloneqq \left(\int_{\Omega} f(x)^2 \diff \mu(x) \right)^{1/2} < \infty$.
    \end{itemize}
\end{definition}

\begin{proposition}
    Ein kontinuierliches Energie-Signal mit kompaktem Träger ist $L^1$-endlich.
    Außerdem sind diskrete Signale mit kompaktem Träger $L^1$-endlich.
\end{proposition}
\begin{proof}
    Sei $f$ ein kontinuierliches Energie-Signal mit $f(x) = 0$ für $\norm{x}_{\infty} \geq R$.
    Sei $\mathbb{1}_R$ die charakteristische Funktion mit $\mathbb{1}_R(x) = 1$ für $\norm{x}_\infty \leq R$ und $\mathbb{1}_R(x) = 0$ sonst.
    Sei außerdem $\langle \,\cdot\, , \,\cdot\, \rangle_{L_2}$ das Skalarprodukt auf dem Hilbertraum $L^2$.
    Dann gilt mit der Cauchy-Schwarz-Ungleichung
    \[
        \norm{f}_1 
        = \int_{\R^d} \abs{f(x)} \cdot \mathbb{1}_R(x) \diff x
        = \langle \abs{f}, \mathbb{1}_R \rangle_{L^2}
        \leq \norm{\abs{f}}_2 \cdot \norm{\mathbb{1}_R}_2 < \infty .
    \]

    Für diskrete Signale mit kompaktem Träger folgt die $L^1$-Endlichkeit bereits aus der Endlichkeit der Menge $\{ z\in\Z^d \mid \norm{z}_{\infty} \leq R\}$.
\end{proof}

\begin{definition}
    Die \emph{Faltung} zweier diskreter bzw. kontinuierlicher Signale $f,g: \Omega\rightarrow \R$ ist definiert als 
    \[
        (f * g)(x) \coloneqq \int_\Omega f(\tau) \cdot g(x-\tau) \diff \mu(\tau).
    \]
    Die \emph{Kreuzkorrelation} von $f$ und $g$ ist definiert als
    \[
        (f \star g)(x) \coloneqq \int_\Omega f(\tau) \cdot g(x+\tau) \diff \mu(\tau).
    \]
    Dabei wird $f$ meist der \emph{Kern} oder \emph{Filter} und $g$ das \emph{(Eingangs-)Signal} genannt.
\end{definition}

Wir bemerken, dass der einzige Unterschied der Operationen darin besteht, dass bei der Faltung der Kern \glqq geflippt\grqq\ wird.
Es gilt nämlich \[
    \left(f \star g\right)(x) = \int_{\Omega} f(\tau) \cdot g(x+\tau) \diff\mu(\tau) = \int_\Omega f(-\tau) \cdot g(x - \tau) \diff\mu(\tau) = \left((\tau\mapsto f(-\tau)) * g\right)(x).
    \]
Ist der Kern \emph{symmetrisch}, gilt also $f(x) = f(-x)$ für alle $x\in\Omega$, so stimmt Kreuzkorrelation mit Faltung überein.
Durch das \glqq Flippen\grqq\ des Kerns wird die Faltungsoperation kommutativ, was für die Kreuzkorrelation nicht gilt.
Oft wird die Faltung bzw. Kreuzkorrelation so dargestellt, dass der Kern diejenige Funktion ist, welche verschoben und gegebenenfalls geflippt wird:
\[
    (f*g)(x) = \int_\Omega f(x - \tau) \cdot g(\tau)\diff\mu(\tau), \quad
    (f\star g)(x) = \int_\Omega f(\tau - x) \cdot g(\tau) \diff\mu(\tau).
\]

Im Deep-Learning-Kontext wird meist von Faltung gesprochen, dabei wird aber tatsächlich die Kreuzkorrelation gemeint und verwendet.
\todo{Später werden wir sehen, warum dies keine größeren Auswirkungen auf die Neuronalen Faltungslayer hat.}

Wir betrachten zunächst einige Beispiele dieser Operationen.
Definiere dazu die Signale
\begin{align*}
    f: \R \rightarrow \R, \quad x \mapsto \begin{cases}
        \frac{1}{2} - x, & \text{für $x\in[-\frac{1}{2},\frac{1}{2}]$,}\\
        0, & \text{sonst,}
    \end{cases} \qquad
    g: \R \rightarrow \R,\quad x \mapsto \begin{cases}
        1,& \text{für $x \in [-\frac{1}{2},\frac{1}{2}]$}\\
        0, & \text{sonst.}
    \end{cases}
\end{align*}

In Abbildung~\todo{figure-a} kann man die Faltung $f * g$ der beiden Signale sehen, in Abbildung~\todo{figure-b} die Kreuzkorrelation~$f\star g$.


In Anwendungen werden meist diskrete Signale mit kompaktem Träger betrachtet.
Ein diskretes Signal dieser Form schreiben wir im eindimensionalen Fall $d=1$ als Vektor $f = (f_0, f_1, \dots, f_{n-1})$ und im zweidimensionalen Fall $d=2$ als Matrix $$F = (f_{i,j})_{\substack{i=0,\dots,m-1\\ j=0,\dots, n-1}}.$$
Diese Signale werden für Indizes außerhalb der so-definierten Vektoren bzw. Matrizen als $0$ ausgewertet, also beispielsweise $f_z = 0$ für $z\in\Z\setminus[0,n-1]$ mit $f=(f_0, \dots, f_{n-1})$.

Betrachtet man die Kreuzkorrelation von $g$ mit dem Filter $f=(\frac{1}{2}, \frac{1}{2})$, so erhält man den sogenannten \emph{gleitenden Mittelwert} von $g$:
\[
    (f \star g)_i = \int_\Z f_\tau \cdot g_{i + \tau} \diff \cmeasure(\tau) 
    = \sum_{\tau\in\Z} f_\tau \cdot g_{i+ \tau}
    = \frac{g_i}{2} + \frac{g_{i+1}}{2} = \frac{g_i + g_{i+1}}{2}
\]

Der sogenannte Sobel-Filter zur Erkennung von horizontalen Kanten hat die Form
\[
    F = \begin{pmatrix}
        1 & 2 & 1 \\
        0 & 0 & 0 \\ 
        -1 & -2 & -1
    \end{pmatrix}.
\]




\begin{figure}
    \begin{subfigure}[c]{0.5\textwidth}
        \resizebox{\textwidth}{\textwidth}{
        
\tikzset{ 
table/.style={
    matrix of nodes,
    nodes={rectangle,minimum size=0.05cm,align=center},
    nodes in empty cells
    }
}
\begin{tikzpicture}
    \matrix (mat) [table]
    {
    |[fill=white!0!black] | & |[fill=white!0!black] | & |[fill=white!0!black] | & |[fill=white!0!black] | & |[fill=white!0!black] | & |[fill=white!0!black] | & |[fill=white!0!black] | & |[fill=white!0!black] | & |[fill=white!0!black] | & |[fill=white!0!black] | & |[fill=white!0!black] | & |[fill=white!0!black] | & |[fill=white!0!black] | & |[fill=white!0!black] |\\
    |[fill=white!0!black] | & |[fill=white!0!black] | & |[fill=white!0!black] | & |[fill=white!0!black] | & |[fill=white!0!black] | & |[fill=white!0!black] | & |[fill=white!0!black] | & |[fill=white!0!black] | & |[fill=white!0!black] | & |[fill=white!0!black] | & |[fill=white!0!black] | & |[fill=white!0!black] | & |[fill=white!0!black] | & |[fill=white!0!black] |\\
    |[fill=white!0!black] | & |[fill=white!0!black] | & |[fill=white!0!black] | & |[fill=white!0!black] | & |[fill=white!0!black] | & |[fill=white!0!black] | & |[fill=white!0!black] | & |[fill=white!0!black] | & |[fill=white!0!black] | & |[fill=white!0!black] | & |[fill=white!110!black] | & |[fill=white!103!black] | & |[fill=white!0!black] | & |[fill=white!0!black] |\\
    |[fill=white!0!black] | & |[fill=white!0!black] | & |[fill=white!0!black] | & |[fill=white!0!black] | & |[fill=white!0!black] | & |[fill=white!0!black] | & |[fill=white!0!black] | & |[fill=white!0!black] | & |[fill=white!0!black] | & |[fill=white!15!black] | & |[fill=white!238!black] | & |[fill=white!62!black] | & |[fill=white!0!black] | & |[fill=white!0!black] |\\
    |[fill=white!0!black] | & |[fill=white!0!black] | & |[fill=white!0!black] | & |[fill=white!0!black] | & |[fill=white!0!black] | & |[fill=white!0!black] | & |[fill=white!0!black] | & |[fill=white!0!black] | & |[fill=white!0!black] | & |[fill=white!138!black] | & |[fill=white!183!black] | & |[fill=white!2!black] | & |[fill=white!0!black] | & |[fill=white!0!black] |\\
    |[fill=white!0!black] | & |[fill=white!0!black] | & |[fill=white!0!black] | & |[fill=white!0!black] | & |[fill=white!0!black] | & |[fill=white!21!black] | & |[fill=white!134!black] | & |[fill=white!5!black] | & |[fill=white!51!black] | & |[fill=white!233!black] | & |[fill=white!39!black] | & |[fill=white!0!black] | & |[fill=white!0!black] | & |[fill=white!0!black] |\\
    |[fill=white!0!black] | & |[fill=white!0!black] | & |[fill=white!0!black] | & |[fill=white!0!black] | & |[fill=white!0!black] | & |[fill=white!24!black] | & |[fill=white!134!black] | & |[fill=white!10!black] | & |[fill=white!204!black] | & |[fill=white!133!black] | & |[fill=white!0!black] | & |[fill=white!0!black] | & |[fill=white!0!black] | & |[fill=white!0!black] |\\
    |[fill=white!0!black] | & |[fill=white!0!black] | & |[fill=white!0!black] | & |[fill=white!0!black] | & |[fill=white!91!black] | & |[fill=white!209!black] | & |[fill=white!19!black] | & |[fill=white!108!black] | & |[fill=white!197!black] | & |[fill=white!5!black] | & |[fill=white!0!black] | & |[fill=white!0!black] | & |[fill=white!0!black] | & |[fill=white!0!black] |\\
    |[fill=white!0!black] | & |[fill=white!0!black] | & |[fill=white!0!black] | & |[fill=white!73!black] | & |[fill=white!248!black] | & |[fill=white!95!black] | & |[fill=white!76!black] | & |[fill=white!222!black] | & |[fill=white!119!black] | & |[fill=white!66!black] | & |[fill=white!0!black] | & |[fill=white!0!black] | & |[fill=white!0!black] | & |[fill=white!0!black] |\\
    |[fill=white!0!black] | & |[fill=white!0!black] | & |[fill=white!0!black] | & |[fill=white!81!black] | & |[fill=white!182!black] | & |[fill=white!209!black] | & |[fill=white!254!black] | & |[fill=white!163!black] | & |[fill=white!84!black] | & |[fill=white!16!black] | & |[fill=white!0!black] | & |[fill=white!0!black] | & |[fill=white!0!black] | & |[fill=white!0!black] |\\
    |[fill=white!0!black] | & |[fill=white!0!black] | & |[fill=white!0!black] | & |[fill=white!0!black] | & |[fill=white!0!black] | & |[fill=white!104!black] | & |[fill=white!161!black] | & |[fill=white!0!black] | & |[fill=white!0!black] | & |[fill=white!0!black] | & |[fill=white!0!black] | & |[fill=white!0!black] | & |[fill=white!0!black] | & |[fill=white!0!black] |\\
    |[fill=white!0!black] | & |[fill=white!0!black] | & |[fill=white!0!black] | & |[fill=white!0!black] | & |[fill=white!0!black] | & |[fill=white!210!black] | & |[fill=white!37!black] | & |[fill=white!0!black] | & |[fill=white!0!black] | & |[fill=white!0!black] | & |[fill=white!0!black] | & |[fill=white!0!black] | & |[fill=white!0!black] | & |[fill=white!0!black] |\\
    |[fill=white!0!black] | & |[fill=white!0!black] | & |[fill=white!0!black] | & |[fill=white!0!black] | & |[fill=white!0!black] | & |[fill=white!0!black] | & |[fill=white!0!black] | & |[fill=white!0!black] | & |[fill=white!0!black] | & |[fill=white!0!black] | & |[fill=white!0!black] | & |[fill=white!0!black] | & |[fill=white!0!black] | & |[fill=white!0!black] |\\
    |[fill=white!0!black] | & |[fill=white!0!black] | & |[fill=white!0!black] | & |[fill=white!0!black] | & |[fill=white!0!black] | & |[fill=white!0!black] | & |[fill=white!0!black] | & |[fill=white!0!black] | & |[fill=white!0!black] | & |[fill=white!0!black] | & |[fill=white!0!black] | & |[fill=white!0!black] | & |[fill=white!0!black] | & |[fill=white!0!black] |\\
    };
\end{tikzpicture}
        }
        \captionsetup{justification=centering}
        \caption{Das Inputsignal $G$.\\Schwarz $\hat=$ $0$, Weiß $\hat=$ $1$}
    \end{subfigure}%
    \begin{subfigure}[c]{0.5\textwidth}
        \resizebox{\textwidth}{\textwidth}{
        
\tikzset{ 
table/.style={
    matrix of nodes,
    nodes={rectangle,minimum size=0.05cm,align=center},
    nodes in empty cells
    }
}
\begin{tikzpicture}
    \matrix (mat) [table]
    {
    |[fill=white!0!black] | & |[fill=white!0!black] | & |[fill=white!0!black] | & |[fill=white!0!black] | & |[fill=white!0!black] | & |[fill=white!0!black] | & |[fill=white!0!black] | & |[fill=white!0!black] | & |[fill=white!0!black] | & |[fill=white!0!black] | & |[fill=white!0!black] | & |[fill=white!0!black] | & |[fill=white!0!black] | & |[fill=white!0!black] | & |[fill=white!0!black] | & |[fill=white!0!black] | & |[fill=white!0!black] | & |[fill=white!0!black] | & |[fill=white!0!black] | & |[fill=white!0!black] | & |[fill=white!0!black] | & |[fill=white!0!black] | & |[fill=white!0!black] | & |[fill=white!0!black] | & |[fill=white!0!black] | & |[fill=white!0!black] | & |[fill=white!0!black] | & |[fill=white!0!black] |\\
    |[fill=white!0!black] | & |[fill=white!0!black] | & |[fill=white!0!black] | & |[fill=white!0!black] | & |[fill=white!0!black] | & |[fill=white!0!black] | & |[fill=white!0!black] | & |[fill=white!0!black] | & |[fill=white!0!black] | & |[fill=white!0!black] | & |[fill=white!0!black] | & |[fill=white!0!black] | & |[fill=white!0!black] | & |[fill=white!0!black] | & |[fill=white!0!black] | & |[fill=white!0!black] | & |[fill=white!0!black] | & |[fill=white!0!black] | & |[fill=white!0!black] | & |[fill=white!0!black] | & |[fill=white!0!black] | & |[fill=white!0!black] | & |[fill=white!0!black] | & |[fill=white!0!black] | & |[fill=white!0!black] | & |[fill=white!0!black] | & |[fill=white!0!black] | & |[fill=white!0!black] |\\
    |[fill=white!0!black] | & |[fill=white!0!black] | & |[fill=white!0!black] | & |[fill=white!0!black] | & |[fill=white!0!black] | & |[fill=white!0!black] | & |[fill=white!0!black] | & |[fill=white!0!black] | & |[fill=white!0!black] | & |[fill=white!0!black] | & |[fill=white!0!black] | & |[fill=white!0!black] | & |[fill=white!0!black] | & |[fill=white!0!black] | & |[fill=white!0!black] | & |[fill=white!0!black] | & |[fill=white!0!black] | & |[fill=white!0!black] | & |[fill=white!0!black] | & |[fill=white!0!black] | & |[fill=white!0!black] | & |[fill=white!0!black] | & |[fill=white!0!black] | & |[fill=white!0!black] | & |[fill=white!0!black] | & |[fill=white!0!black] | & |[fill=white!0!black] | & |[fill=white!0!black] |\\
    |[fill=white!0!black] | & |[fill=white!0!black] | & |[fill=white!0!black] | & |[fill=white!0!black] | & |[fill=white!0!black] | & |[fill=white!0!black] | & |[fill=white!0!black] | & |[fill=white!0!black] | & |[fill=white!0!black] | & |[fill=white!0!black] | & |[fill=white!0!black] | & |[fill=white!0!black] | & |[fill=white!0!black] | & |[fill=white!0!black] | & |[fill=white!0!black] | & |[fill=white!0!black] | & |[fill=white!0!black] | & |[fill=white!0!black] | & |[fill=white!0!black] | & |[fill=white!0!black] | & |[fill=white!0!black] | & |[fill=white!0!black] | & |[fill=white!0!black] | & |[fill=white!0!black] | & |[fill=white!0!black] | & |[fill=white!0!black] | & |[fill=white!0!black] | & |[fill=white!0!black] |\\
    |[fill=white!0!black] | & |[fill=white!0!black] | & |[fill=white!0!black] | & |[fill=white!0!black] | & |[fill=white!0!black] | & |[fill=white!0!black] | & |[fill=white!0!black] | & |[fill=white!0!black] | & |[fill=white!0!black] | & |[fill=white!0!black] | & |[fill=white!0!black] | & |[fill=white!0!black] | & |[fill=white!0!black] | & |[fill=white!0!black] | & |[fill=white!0!black] | & |[fill=white!0!black] | & |[fill=white!0!black] | & |[fill=white!0!black] | & |[fill=white!0!black] | & |[fill=white!0!black] | & |[fill=white!0!black] | & |[fill=white!0!black] | & |[fill=white!0!black] | & |[fill=white!0!black] | & |[fill=white!0!black] | & |[fill=white!0!black] | & |[fill=white!0!black] | & |[fill=white!0!black] |\\
    |[fill=white!0!black] | & |[fill=white!0!black] | & |[fill=white!0!black] | & |[fill=white!0!black] | & |[fill=white!0!black] | & |[fill=white!0!black] | & |[fill=white!0!black] | & |[fill=white!0!black] | & |[fill=white!0!black] | & |[fill=white!0!black] | & |[fill=white!0!black] | & |[fill=white!0!black] | & |[fill=white!0!black] | & |[fill=white!0!black] | & |[fill=white!0!black] | & |[fill=white!0!black] | & |[fill=white!0!black] | & |[fill=white!0!black] | & |[fill=white!0!black] | & |[fill=white!0!black] | & |[fill=white!0!black] | & |[fill=white!0!black] | & |[fill=white!0!black] | & |[fill=white!3!black] | & |[fill=white!0!black] | & |[fill=white!0!black] | & |[fill=white!0!black] | & |[fill=white!0!black] |\\
    |[fill=white!0!black] | & |[fill=white!0!black] | & |[fill=white!0!black] | & |[fill=white!0!black] | & |[fill=white!0!black] | & |[fill=white!0!black] | & |[fill=white!0!black] | & |[fill=white!0!black] | & |[fill=white!0!black] | & |[fill=white!0!black] | & |[fill=white!0!black] | & |[fill=white!0!black] | & |[fill=white!0!black] | & |[fill=white!0!black] | & |[fill=white!0!black] | & |[fill=white!0!black] | & |[fill=white!0!black] | & |[fill=white!0!black] | & |[fill=white!0!black] | & |[fill=white!0!black] | & |[fill=white!0!black] | & |[fill=white!0!black] | & |[fill=white!47!black] | & |[fill=white!53!black] | & |[fill=white!0!black] | & |[fill=white!0!black] | & |[fill=white!0!black] | & |[fill=white!0!black] |\\
    |[fill=white!0!black] | & |[fill=white!0!black] | & |[fill=white!0!black] | & |[fill=white!0!black] | & |[fill=white!0!black] | & |[fill=white!0!black] | & |[fill=white!0!black] | & |[fill=white!0!black] | & |[fill=white!0!black] | & |[fill=white!0!black] | & |[fill=white!0!black] | & |[fill=white!0!black] | & |[fill=white!0!black] | & |[fill=white!0!black] | & |[fill=white!0!black] | & |[fill=white!0!black] | & |[fill=white!0!black] | & |[fill=white!0!black] | & |[fill=white!0!black] | & |[fill=white!0!black] | & |[fill=white!0!black] | & |[fill=white!57!black] | & |[fill=white!78!black] | & |[fill=white!63!black] | & |[fill=white!0!black] | & |[fill=white!0!black] | & |[fill=white!0!black] | & |[fill=white!0!black] |\\
    |[fill=white!0!black] | & |[fill=white!0!black] | & |[fill=white!0!black] | & |[fill=white!0!black] | & |[fill=white!0!black] | & |[fill=white!0!black] | & |[fill=white!0!black] | & |[fill=white!0!black] | & |[fill=white!0!black] | & |[fill=white!0!black] | & |[fill=white!0!black] | & |[fill=white!0!black] | & |[fill=white!0!black] | & |[fill=white!0!black] | & |[fill=white!0!black] | & |[fill=white!0!black] | & |[fill=white!0!black] | & |[fill=white!0!black] | & |[fill=white!0!black] | & |[fill=white!0!black] | & |[fill=white!14!black] | & |[fill=white!95!black] | & |[fill=white!84!black] | & |[fill=white!16!black] | & |[fill=white!0!black] | & |[fill=white!0!black] | & |[fill=white!0!black] | & |[fill=white!0!black] |\\
    |[fill=white!0!black] | & |[fill=white!0!black] | & |[fill=white!0!black] | & |[fill=white!0!black] | & |[fill=white!0!black] | & |[fill=white!0!black] | & |[fill=white!0!black] | & |[fill=white!0!black] | & |[fill=white!0!black] | & |[fill=white!0!black] | & |[fill=white!0!black] | & |[fill=white!0!black] | & |[fill=white!0!black] | & |[fill=white!0!black] | & |[fill=white!0!black] | & |[fill=white!0!black] | & |[fill=white!0!black] | & |[fill=white!0!black] | & |[fill=white!0!black] | & |[fill=white!0!black] | & |[fill=white!88!black] | & |[fill=white!113!black] | & |[fill=white!73!black] | & |[fill=white!3!black] | & |[fill=white!0!black] | & |[fill=white!0!black] | & |[fill=white!0!black] | & |[fill=white!0!black] |\\
    |[fill=white!0!black] | & |[fill=white!0!black] | & |[fill=white!0!black] | & |[fill=white!0!black] | & |[fill=white!0!black] | & |[fill=white!0!black] | & |[fill=white!0!black] | & |[fill=white!0!black] | & |[fill=white!0!black] | & |[fill=white!0!black] | & |[fill=white!0!black] | & |[fill=white!0!black] | & |[fill=white!0!black] | & |[fill=white!0!black] | & |[fill=white!0!black] | & |[fill=white!0!black] | & |[fill=white!0!black] | & |[fill=white!0!black] | & |[fill=white!0!black] | & |[fill=white!32!black] | & |[fill=white!100!black] | & |[fill=white!83!black] | & |[fill=white!17!black] | & |[fill=white!0!black] | & |[fill=white!0!black] | & |[fill=white!0!black] | & |[fill=white!0!black] | & |[fill=white!0!black] |\\
    |[fill=white!0!black] | & |[fill=white!0!black] | & |[fill=white!0!black] | & |[fill=white!0!black] | & |[fill=white!0!black] | & |[fill=white!0!black] | & |[fill=white!0!black] | & |[fill=white!0!black] | & |[fill=white!0!black] | & |[fill=white!0!black] | & |[fill=white!0!black] | & |[fill=white!0!black] | & |[fill=white!0!black] | & |[fill=white!0!black] | & |[fill=white!0!black] | & |[fill=white!0!black] | & |[fill=white!0!black] | & |[fill=white!0!black] | & |[fill=white!0!black] | & |[fill=white!82!black] | & |[fill=white!92!black] | & |[fill=white!45!black] | & |[fill=white!0!black] | & |[fill=white!0!black] | & |[fill=white!0!black] | & |[fill=white!0!black] | & |[fill=white!0!black] | & |[fill=white!0!black] |\\
    |[fill=white!0!black] | & |[fill=white!0!black] | & |[fill=white!0!black] | & |[fill=white!0!black] | & |[fill=white!0!black] | & |[fill=white!0!black] | & |[fill=white!0!black] | & |[fill=white!0!black] | & |[fill=white!0!black] | & |[fill=white!0!black] | & |[fill=white!27!black] | & |[fill=white!75!black] | & |[fill=white!137!black] | & |[fill=white!115!black] | & |[fill=white!57!black] | & |[fill=white!0!black] | & |[fill=white!0!black] | & |[fill=white!0!black] | & |[fill=white!58!black] | & |[fill=white!103!black] | & |[fill=white!72!black] | & |[fill=white!6!black] | & |[fill=white!0!black] | & |[fill=white!0!black] | & |[fill=white!0!black] | & |[fill=white!0!black] | & |[fill=white!0!black] | & |[fill=white!0!black] |\\
    |[fill=white!0!black] | & |[fill=white!0!black] | & |[fill=white!0!black] | & |[fill=white!0!black] | & |[fill=white!0!black] | & |[fill=white!0!black] | & |[fill=white!0!black] | & |[fill=white!0!black] | & |[fill=white!0!black] | & |[fill=white!0!black] | & |[fill=white!0!black] | & |[fill=white!0!black] | & |[fill=white!100!black] | & |[fill=white!127!black] | & |[fill=white!1!black] | & |[fill=white!0!black] | & |[fill=white!0!black] | & |[fill=white!43!black] | & |[fill=white!134!black] | & |[fill=white!115!black] | & |[fill=white!38!black] | & |[fill=white!0!black] | & |[fill=white!0!black] | & |[fill=white!0!black] | & |[fill=white!0!black] | & |[fill=white!0!black] | & |[fill=white!0!black] | & |[fill=white!0!black] |\\
    |[fill=white!0!black] | & |[fill=white!0!black] | & |[fill=white!0!black] | & |[fill=white!0!black] | & |[fill=white!0!black] | & |[fill=white!0!black] | & |[fill=white!0!black] | & |[fill=white!0!black] | & |[fill=white!0!black] | & |[fill=white!0!black] | & |[fill=white!0!black] | & |[fill=white!0!black] | & |[fill=white!0!black] | & |[fill=white!12!black] | & |[fill=white!0!black] | & |[fill=white!0!black] | & |[fill=white!0!black] | & |[fill=white!97!black] | & |[fill=white!100!black] | & |[fill=white!55!black] | & |[fill=white!1!black] | & |[fill=white!0!black] | & |[fill=white!0!black] | & |[fill=white!0!black] | & |[fill=white!0!black] | & |[fill=white!0!black] | & |[fill=white!0!black] | & |[fill=white!0!black] |\\
    |[fill=white!0!black] | & |[fill=white!0!black] | & |[fill=white!0!black] | & |[fill=white!0!black] | & |[fill=white!0!black] | & |[fill=white!0!black] | & |[fill=white!0!black] | & |[fill=white!0!black] | & |[fill=white!0!black] | & |[fill=white!0!black] | & |[fill=white!0!black] | & |[fill=white!60!black] | & |[fill=white!54!black] | & |[fill=white!0!black] | & |[fill=white!0!black] | & |[fill=white!0!black] | & |[fill=white!61!black] | & |[fill=white!101!black] | & |[fill=white!73!black] | & |[fill=white!7!black] | & |[fill=white!0!black] | & |[fill=white!0!black] | & |[fill=white!0!black] | & |[fill=white!0!black] | & |[fill=white!0!black] | & |[fill=white!0!black] | & |[fill=white!0!black] | & |[fill=white!0!black] |\\
    |[fill=white!0!black] | & |[fill=white!0!black] | & |[fill=white!0!black] | & |[fill=white!0!black] | & |[fill=white!0!black] | & |[fill=white!0!black] | & |[fill=white!0!black] | & |[fill=white!0!black] | & |[fill=white!0!black] | & |[fill=white!0!black] | & |[fill=white!92!black] | & |[fill=white!72!black] | & |[fill=white!0!black] | & |[fill=white!0!black] | & |[fill=white!0!black] | & |[fill=white!0!black] | & |[fill=white!0!black] | & |[fill=white!0!black] | & |[fill=white!0!black] | & |[fill=white!0!black] | & |[fill=white!0!black] | & |[fill=white!0!black] | & |[fill=white!0!black] | & |[fill=white!0!black] | & |[fill=white!0!black] | & |[fill=white!0!black] | & |[fill=white!0!black] | & |[fill=white!0!black] |\\
    |[fill=white!0!black] | & |[fill=white!0!black] | & |[fill=white!0!black] | & |[fill=white!0!black] | & |[fill=white!0!black] | & |[fill=white!0!black] | & |[fill=white!0!black] | & |[fill=white!0!black] | & |[fill=white!0!black] | & |[fill=white!0!black] | & |[fill=white!0!black] | & |[fill=white!0!black] | & |[fill=white!0!black] | & |[fill=white!0!black] | & |[fill=white!0!black] | & |[fill=white!0!black] | & |[fill=white!0!black] | & |[fill=white!0!black] | & |[fill=white!0!black] | & |[fill=white!0!black] | & |[fill=white!0!black] | & |[fill=white!0!black] | & |[fill=white!0!black] | & |[fill=white!0!black] | & |[fill=white!0!black] | & |[fill=white!0!black] | & |[fill=white!0!black] | & |[fill=white!0!black] |\\
    |[fill=white!0!black] | & |[fill=white!0!black] | & |[fill=white!0!black] | & |[fill=white!0!black] | & |[fill=white!0!black] | & |[fill=white!9!black] | & |[fill=white!62!black] | & |[fill=white!115!black] | & |[fill=white!148!black] | & |[fill=white!63!black] | & |[fill=white!0!black] | & |[fill=white!0!black] | & |[fill=white!0!black] | & |[fill=white!0!black] | & |[fill=white!83!black] | & |[fill=white!170!black] | & |[fill=white!183!black] | & |[fill=white!157!black] | & |[fill=white!134!black] | & |[fill=white!88!black] | & |[fill=white!34!black] | & |[fill=white!0!black] | & |[fill=white!0!black] | & |[fill=white!0!black] | & |[fill=white!0!black] | & |[fill=white!0!black] | & |[fill=white!0!black] | & |[fill=white!0!black] |\\
    |[fill=white!0!black] | & |[fill=white!0!black] | & |[fill=white!0!black] | & |[fill=white!0!black] | & |[fill=white!0!black] | & |[fill=white!9!black] | & |[fill=white!86!black] | & |[fill=white!170!black] | & |[fill=white!246!black] | & |[fill=white!253!black] | & |[fill=white!206!black] | & |[fill=white!121!black] | & |[fill=white!68!black] | & |[fill=white!115!black] | & |[fill=white!194!black] | & |[fill=white!228!black] | & |[fill=white!189!black] | & |[fill=white!132!black] | & |[fill=white!67!black] | & |[fill=white!21!black] | & |[fill=white!0!black] | & |[fill=white!0!black] | & |[fill=white!0!black] | & |[fill=white!0!black] | & |[fill=white!0!black] | & |[fill=white!0!black] | & |[fill=white!0!black] | & |[fill=white!0!black] |\\
    |[fill=white!0!black] | & |[fill=white!0!black] | & |[fill=white!0!black] | & |[fill=white!0!black] | & |[fill=white!0!black] | & |[fill=white!0!black] | & |[fill=white!22!black] | & |[fill=white!53!black] | & |[fill=white!96!black] | & |[fill=white!113!black] | & |[fill=white!61!black] | & |[fill=white!29!black] | & |[fill=white!72!black] | & |[fill=white!143!black] | & |[fill=white!136!black] | & |[fill=white!55!black] | & |[fill=white!4!black] | & |[fill=white!0!black] | & |[fill=white!0!black] | & |[fill=white!0!black] | & |[fill=white!0!black] | & |[fill=white!0!black] | & |[fill=white!0!black] | & |[fill=white!0!black] | & |[fill=white!0!black] | & |[fill=white!0!black] | & |[fill=white!0!black] | & |[fill=white!0!black] |\\
    |[fill=white!0!black] | & |[fill=white!0!black] | & |[fill=white!0!black] | & |[fill=white!0!black] | & |[fill=white!0!black] | & |[fill=white!0!black] | & |[fill=white!0!black] | & |[fill=white!0!black] | & |[fill=white!0!black] | & |[fill=white!0!black] | & |[fill=white!0!black] | & |[fill=white!0!black] | & |[fill=white!78!black] | & |[fill=white!115!black] | & |[fill=white!53!black] | & |[fill=white!0!black] | & |[fill=white!0!black] | & |[fill=white!0!black] | & |[fill=white!0!black] | & |[fill=white!0!black] | & |[fill=white!0!black] | & |[fill=white!0!black] | & |[fill=white!0!black] | & |[fill=white!0!black] | & |[fill=white!0!black] | & |[fill=white!0!black] | & |[fill=white!0!black] | & |[fill=white!0!black] |\\
    |[fill=white!0!black] | & |[fill=white!0!black] | & |[fill=white!0!black] | & |[fill=white!0!black] | & |[fill=white!0!black] | & |[fill=white!0!black] | & |[fill=white!0!black] | & |[fill=white!0!black] | & |[fill=white!0!black] | & |[fill=white!0!black] | & |[fill=white!0!black] | & |[fill=white!3!black] | & |[fill=white!46!black] | & |[fill=white!50!black] | & |[fill=white!3!black] | & |[fill=white!0!black] | & |[fill=white!0!black] | & |[fill=white!0!black] | & |[fill=white!0!black] | & |[fill=white!0!black] | & |[fill=white!0!black] | & |[fill=white!0!black] | & |[fill=white!0!black] | & |[fill=white!0!black] | & |[fill=white!0!black] | & |[fill=white!0!black] | & |[fill=white!0!black] | & |[fill=white!0!black] |\\
    |[fill=white!0!black] | & |[fill=white!0!black] | & |[fill=white!0!black] | & |[fill=white!0!black] | & |[fill=white!0!black] | & |[fill=white!0!black] | & |[fill=white!0!black] | & |[fill=white!0!black] | & |[fill=white!0!black] | & |[fill=white!60!black] | & |[fill=white!144!black] | & |[fill=white!167!black] | & |[fill=white!107!black] | & |[fill=white!23!black] | & |[fill=white!0!black] | & |[fill=white!0!black] | & |[fill=white!0!black] | & |[fill=white!0!black] | & |[fill=white!0!black] | & |[fill=white!0!black] | & |[fill=white!0!black] | & |[fill=white!0!black] | & |[fill=white!0!black] | & |[fill=white!0!black] | & |[fill=white!0!black] | & |[fill=white!0!black] | & |[fill=white!0!black] | & |[fill=white!0!black] |\\
    |[fill=white!0!black] | & |[fill=white!0!black] | & |[fill=white!0!black] | & |[fill=white!0!black] | & |[fill=white!0!black] | & |[fill=white!0!black] | & |[fill=white!0!black] | & |[fill=white!0!black] | & |[fill=white!0!black] | & |[fill=white!54!black] | & |[fill=white!136!black] | & |[fill=white!162!black] | & |[fill=white!107!black] | & |[fill=white!26!black] | & |[fill=white!0!black] | & |[fill=white!0!black] | & |[fill=white!0!black] | & |[fill=white!0!black] | & |[fill=white!0!black] | & |[fill=white!0!black] | & |[fill=white!0!black] | & |[fill=white!0!black] | & |[fill=white!0!black] | & |[fill=white!0!black] | & |[fill=white!0!black] | & |[fill=white!0!black] | & |[fill=white!0!black] | & |[fill=white!0!black] |\\
    |[fill=white!0!black] | & |[fill=white!0!black] | & |[fill=white!0!black] | & |[fill=white!0!black] | & |[fill=white!0!black] | & |[fill=white!0!black] | & |[fill=white!0!black] | & |[fill=white!0!black] | & |[fill=white!0!black] | & |[fill=white!0!black] | & |[fill=white!0!black] | & |[fill=white!0!black] | & |[fill=white!0!black] | & |[fill=white!0!black] | & |[fill=white!0!black] | & |[fill=white!0!black] | & |[fill=white!0!black] | & |[fill=white!0!black] | & |[fill=white!0!black] | & |[fill=white!0!black] | & |[fill=white!0!black] | & |[fill=white!0!black] | & |[fill=white!0!black] | & |[fill=white!0!black] | & |[fill=white!0!black] | & |[fill=white!0!black] | & |[fill=white!0!black] | & |[fill=white!0!black] |\\
    |[fill=white!0!black] | & |[fill=white!0!black] | & |[fill=white!0!black] | & |[fill=white!0!black] | & |[fill=white!0!black] | & |[fill=white!0!black] | & |[fill=white!0!black] | & |[fill=white!0!black] | & |[fill=white!0!black] | & |[fill=white!0!black] | & |[fill=white!0!black] | & |[fill=white!0!black] | & |[fill=white!0!black] | & |[fill=white!0!black] | & |[fill=white!0!black] | & |[fill=white!0!black] | & |[fill=white!0!black] | & |[fill=white!0!black] | & |[fill=white!0!black] | & |[fill=white!0!black] | & |[fill=white!0!black] | & |[fill=white!0!black] | & |[fill=white!0!black] | & |[fill=white!0!black] | & |[fill=white!0!black] | & |[fill=white!0!black] | & |[fill=white!0!black] | & |[fill=white!0!black] |\\
    |[fill=white!0!black] | & |[fill=white!0!black] | & |[fill=white!0!black] | & |[fill=white!0!black] | & |[fill=white!0!black] | & |[fill=white!0!black] | & |[fill=white!0!black] | & |[fill=white!0!black] | & |[fill=white!0!black] | & |[fill=white!0!black] | & |[fill=white!0!black] | & |[fill=white!0!black] | & |[fill=white!0!black] | & |[fill=white!0!black] | & |[fill=white!0!black] | & |[fill=white!0!black] | & |[fill=white!0!black] | & |[fill=white!0!black] | & |[fill=white!0!black] | & |[fill=white!0!black] | & |[fill=white!0!black] | & |[fill=white!0!black] | & |[fill=white!0!black] | & |[fill=white!0!black] | & |[fill=white!0!black] | & |[fill=white!0!black] | & |[fill=white!0!black] | & |[fill=white!0!black] |\\
    };
\end{tikzpicture}

        }
    \captionsetup{justification=centering}
    \caption{Die Kreuzkorrelation $F\star G$.\\Schwarz $\hat=$ $-3{,}84$, Weiß $\hat=$ $3{,}98$}
    \end{subfigure}
    \caption{Die Kreuzkorrelation einer handgeschriebenen Ziffer mit dem Sobel-Filter $F$ zur Erkennung von horizontalen Kanten.}
\end{figure}




\section{Faltungsnetze}





\begin{definition}[The Convolution Operation]
    
\end{definition}

\clearpage          % neue Seite für Literaturverzeichnis
\thispagestyle{empty}
\bibliography{../literature}

\end{document}