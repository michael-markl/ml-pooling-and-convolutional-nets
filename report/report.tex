\documentclass{article}
%\usepackage[a5paper,margin=5mm]{geometry}

\usepackage[utf8]{inputenc}
\usepackage[english, ngerman]{babel}
\usepackage{mathtools}
\usepackage{amsmath}
\usepackage{amssymb}
\usepackage{amsthm}
\usepackage{subcaption}
\captionsetup{justification=centering}


\usepackage{mathbbol}
\usepackage{color}

\usepackage{tikz}
\usetikzlibrary{external}
\tikzexternalize
\usetikzlibrary{matrix} 

\DeclareSymbolFontAlphabet{\amsmathbb}{AMSb}%


\bibliographystyle{alphadin}

\newtheorem{theorem}{Theorem}[section]
\newtheorem{proposition}[theorem]{Proposition}
\theoremstyle{definition}
\newtheorem{definition}[theorem]{Definition}
\newcommand{\R}{\mathbb{R}}
\newcommand{\Z}{\mathbb{Z}}
\newcommand{\N}{\mathbb{N}}
\newcommand{\diff}{\,\textrm{d}}
\newcommand{\todo}[1]{{\color{red} #1}}

\newcommand{\norm}[1]{\left\lVert#1\right\rVert}
\newcommand{\abs}[1]{\left\lvert#1\right\rvert}
\newcommand{\avg}{\textnormal{avg}}



\title{Neuronale Faltungsnetze und Pooling}
\author{Michael Markl}

\begin{document}
\maketitle


\section{Einführung}

Neuronale Netze werden bereits heutzutage auf zahlreiche Problemstellungen angewandt.
Dabei ist Computer Vision -- also die Verarbeitung und Analyse von Bildern -- eine der meist genutzten Technologien, die dadurch ermöglicht wird.
Ein klassisches Problem der Computer Vision ist, ein Bild zu einer bestimmten Kategorie zuzuordnen.
Bei der Erkennung von handschriftlichen Ziffern hat LeCun in~\cite{LeCun1989} aufgezeigt, dass das Reduzieren eines Netzwerks durch das Teilen von Parametern mehrere Neuronen zwischen zwei Layern die topologische 2D-Struktur von Bildern ausnutzen kann und dadurch die Klassifizierung erheblich verbessern kann.

% TODO:
Diese Arbeit gibt einen Einblick in Theorie und Praxis von Faltungsnetzen sowie die oft dabei verwendeten Pooling-Layer.
Als Grundlagen werden die Werke von Ian Goodfellow et al. in~\cite[Kapitel~9]{Goodfellow-et-al-2016} sowie von Ovidiu Calin in~\cite[Kapitel~15,16]{Calin2020} herangezogen.

\section{Faltungslayer}

\subsection{Faltung und Kreuzkorrelation}

Bevor wir Faltungsnetze einführen, möchten wir zunächst die zugrundeliegenden mathematischen Konzepte einführen:
Faltung und Kreuzkorrelation.
Diese werden oft in der digitalen Signalverarbeitung genutzt, wo man zwischen kontinuierlichen und diskreten Signalen unterscheidet.

\newcommand{\llambda}{\lambda}
\newcommand{\cmeasure}{\mu_c}

\begin{definition}[Signale]
    Ein \emph{kontinuierliches Signal} ist eine Funktion $f: \R^d\rightarrow \R$.
    Im Kontext von kontinuierlichen Signalen wird das Lebesgue-Maß $\llambda$ auf $\Omega\coloneqq \R^d$ verwendet.

    Ein \emph{diskretes Signal} ist eine Funktion $f: \Z^d \rightarrow \R$.
    Bei diskreten Signalen wird das Zählmaß $\cmeasure$ auf $\Omega\coloneqq \Z^d$ verwendet.

    Ein Signal $f: \Omega \rightarrow \R$
    \begin{itemize}
        \item \emph{hat einen kompakten Träger}, falls es ein $R > 0$ gibt, sodass $f(x) = 0$ für $\norm{x}_\infty \geq R$ gilt,
        \item heißt \emph{$L^1$-endlich}, falls $\norm{f}_1 \coloneqq \int_{\Omega} \abs{f(x)} \diff \mu(x) < \infty$,
        \item heißt \emph{Energie-Signal}, falls $\norm{f}_2 \coloneqq \left(\int_{\Omega} f(x)^2 \diff \mu(x) \right)^{1/2} < \infty$.
    \end{itemize}
\end{definition}

\begin{proposition}
    Ein kontinuierliches Energie-Signal mit kompaktem Träger ist $L^1$-endlich.
    Außerdem sind diskrete Signale mit kompaktem Träger $L^1$-endlich.
\end{proposition}
\begin{proof}
    Sei $f$ ein kontinuierliches Energie-Signal mit $f(x) = 0$ für $\norm{x}_{\infty} \geq R$.
    Sei $\mathbb{1}_R$ die charakteristische Funktion mit $\mathbb{1}_R(x) = 1$ für $\norm{x}_\infty \leq R$ und $\mathbb{1}_R(x) = 0$ sonst.
    Sei außerdem $\langle \,\cdot\, , \,\cdot\, \rangle_{L_2}$ das Skalarprodukt auf dem Hilbertraum $L^2$.
    Dann gilt mit der Cauchy-Schwarz-Ungleichung
    \[
        \norm{f}_1 
        = \int_{\R^d} \abs{f(x)} \cdot \mathbb{1}_R(x) \diff x
        = \langle \abs{f}, \mathbb{1}_R \rangle_{L^2}
        \leq \norm{\abs{f}}_2 \cdot \norm{\mathbb{1}_R}_2 < \infty .
    \]

    Für diskrete Signale mit kompaktem Träger folgt die $L^1$-Endlichkeit bereits aus der Endlichkeit der Menge $\{ z\in\Z^d \mid \norm{z}_{\infty} \leq R\}$.
\end{proof}

\begin{definition}
    Die \emph{Faltung} zweier diskreter bzw. kontinuierlicher Signale $f,g: \Omega\rightarrow \R$ ist definiert als 
    \[
        (f * g)(x) \coloneqq \int_\Omega f(\tau) \cdot g(x-\tau) \diff \mu(\tau).
    \]
    Die \emph{Kreuzkorrelation} von $f$ und $g$ ist definiert als
    \[
        (f \star g)(x) \coloneqq \int_\Omega f(\tau) \cdot g(x+\tau) \diff \mu(\tau).
    \]
    Dabei wird $f$ meist der \emph{Kern} oder \emph{Filter} und $g$ das \emph{(Eingangs-)Signal} genannt.
\end{definition}

Wir bemerken, dass der einzige Unterschied der Operationen darin besteht, dass bei der Faltung der Kern \glqq geflippt\grqq\ wird.
Es gilt nämlich \[
    \left(f \star g\right)(x) = \int_{\Omega} f(\tau) \cdot g(x+\tau) \diff\mu(\tau) = \int_\Omega f(-\tau) \cdot g(x - \tau) \diff\mu(\tau) = \left((\tau\mapsto f(-\tau)) * g\right)(x).
    \]
Ist der Kern \emph{symmetrisch}, gilt also $f(x) = f(-x)$ für alle $x\in\Omega$, so stimmt Kreuzkorrelation mit Faltung überein.
Durch das \glqq Flippen\grqq\ des Kerns wird die Faltungsoperation kommutativ, was für die Kreuzkorrelation nicht gilt.
Oft wird die Faltung bzw. Kreuzkorrelation so dargestellt, dass der Kern diejenige Funktion ist, welche verschoben und gegebenenfalls geflippt wird:
\[
    (f*g)(x) = \int_\Omega f(x - \tau) \cdot g(\tau)\diff\mu(\tau), \quad
    (f\star g)(x) = \int_\Omega f(\tau - x) \cdot g(\tau) \diff\mu(\tau).
\]

\begin{proposition}
    Sind $f$ und $g$ $L^1$-endliche Signale, so auch $f* g$ und $f\star g$.
\end{proposition}
\begin{proof}
    \todo{DO IT! JUST DO IT!}
\end{proof}

Im Deep-Learning-Kontext wird meist von Faltung gesprochen, dabei wird aber tatsächlich oft die Kreuzkorrelation gemeint und verwendet.
\todo{Später werden wir sehen, warum dies keine größeren Auswirkungen auf die Neuronalen Faltungslayer hat.}
Wir werden daher im Folgenden meist nur die Kreuzkorrelation näher betrachten.

Wir betrachten zunächst einige Beispiele dieser Operationen.
Definiere dazu die Signale
\begin{align*}
    f: \R \rightarrow \R, \quad x \mapsto \begin{cases}
        \frac{1}{2} - x, & \text{für $x\in[-\frac{1}{2},\frac{1}{2}]$,}\\
        0, & \text{sonst,}
    \end{cases} \qquad
    g: \R \rightarrow \R,\quad x \mapsto \begin{cases}
        1,& \text{für $x \in [-\frac{1}{2},\frac{1}{2}]$}\\
        0, & \text{sonst.}
    \end{cases}
\end{align*}

In Abbildung~\todo{figure-a} kann man die Faltung $f * g$ der beiden Signale sehen, in Abbildung~\todo{figure-b} die Kreuzkorrelation~$f\star g$.


In Anwendungen werden meist diskrete Signale mit kompaktem Träger betrachtet.
Ein diskretes Signal dieser Form schreiben wir im eindimensionalen Fall $d=1$ als Vektor $f = (f_0, f_1, \dots, f_{n-1})$ und im zweidimensionalen Fall $d=2$ als Matrix $$F = (f_{i,j})_{\substack{i=0,\dots,m-1\\ j=0,\dots, n-1}}.$$
Diese Signale werden für Indizes außerhalb der so-definierten Vektoren bzw. Matrizen als $0$ ausgewertet, also beispielsweise $f_z = 0$ für $z\in\Z\setminus[0,n-1]$ mit $f=(f_0, \dots, f_{n-1})$.

Betrachtet man die Kreuzkorrelation von $g$ mit dem Filter $f=(\frac{1}{2}, \frac{1}{2})$, so erhält man den sogenannten \emph{gleitenden Mittelwert} von $g$:
\[  
    (f \star g)_i = \int_\Z f_\tau \cdot g_{i + \tau} \diff \cmeasure(\tau) 
    = \sum_{\tau\in\Z} f_\tau \cdot g_{i+ \tau}
    = \frac{g_i}{2} + \frac{g_{i+1}}{2} = \frac{g_i + g_{i+1}}{2}
\]

Um Kreuzkorrelation auf Graustufen-Bilder auszuführen, bestimmt man zu\-nächst eine Input-Enkodierung.
Dabei wird oft eine Matrix $G\in[0,1]^{r\times c}$ gewählt, wobei $r$ die Anzahl an Reihen, $c$ die Anzahl an Spalten und $g_{i,j}$ die \emph{Aktivierung des Pixels} zwischen $0$ und $1$ sind, wobei der Wert~$0$ für Schwarz und der Wert~$1$ für Weiß stehen.
Der sogenannte \emph{Sobel-Filter} zur Erkennung von horizontalen Kanten hat die Form
\[
    F = \begin{pmatrix}
        1 & 2 & 1 \\
        0 & 0 & 0 \\ 
        -1 & -2 & -1
    \end{pmatrix}.
\]


\begin{figure}
    \begin{subfigure}{0.5\textwidth}
        \resizebox{\textwidth}{\textwidth}{
        \tikzset{ 
table/.style={
    matrix of nodes,
    nodes={rectangle,minimum size=0.05cm,align=center},
    nodes in empty cells
    }
}
\begin{tikzpicture}
    \matrix (mat) [table]
    {
    |[fill=white!0!black] | & |[fill=white!0!black] | & |[fill=white!0!black] | & |[fill=white!0!black] | & |[fill=white!0!black] | & |[fill=white!0!black] | & |[fill=white!0!black] | & |[fill=white!0!black] | & |[fill=white!0!black] | & |[fill=white!0!black] | & |[fill=white!0!black] | & |[fill=white!0!black] | & |[fill=white!0!black] | & |[fill=white!0!black] | & |[fill=white!0!black] | & |[fill=white!0!black] | & |[fill=white!0!black] | & |[fill=white!0!black] | & |[fill=white!0!black] | & |[fill=white!0!black] | & |[fill=white!0!black] | & |[fill=white!0!black] | & |[fill=white!0!black] | & |[fill=white!0!black] | & |[fill=white!0!black] | & |[fill=white!0!black] | & |[fill=white!0!black] | & |[fill=white!0!black] |\\
    |[fill=white!0!black] | & |[fill=white!0!black] | & |[fill=white!0!black] | & |[fill=white!0!black] | & |[fill=white!0!black] | & |[fill=white!0!black] | & |[fill=white!0!black] | & |[fill=white!0!black] | & |[fill=white!0!black] | & |[fill=white!0!black] | & |[fill=white!0!black] | & |[fill=white!0!black] | & |[fill=white!0!black] | & |[fill=white!0!black] | & |[fill=white!0!black] | & |[fill=white!0!black] | & |[fill=white!0!black] | & |[fill=white!0!black] | & |[fill=white!0!black] | & |[fill=white!0!black] | & |[fill=white!0!black] | & |[fill=white!0!black] | & |[fill=white!0!black] | & |[fill=white!0!black] | & |[fill=white!0!black] | & |[fill=white!0!black] | & |[fill=white!0!black] | & |[fill=white!0!black] |\\
    |[fill=white!0!black] | & |[fill=white!0!black] | & |[fill=white!0!black] | & |[fill=white!0!black] | & |[fill=white!0!black] | & |[fill=white!0!black] | & |[fill=white!0!black] | & |[fill=white!0!black] | & |[fill=white!0!black] | & |[fill=white!0!black] | & |[fill=white!0!black] | & |[fill=white!0!black] | & |[fill=white!0!black] | & |[fill=white!0!black] | & |[fill=white!0!black] | & |[fill=white!0!black] | & |[fill=white!0!black] | & |[fill=white!0!black] | & |[fill=white!0!black] | & |[fill=white!0!black] | & |[fill=white!0!black] | & |[fill=white!0!black] | & |[fill=white!0!black] | & |[fill=white!0!black] | & |[fill=white!0!black] | & |[fill=white!0!black] | & |[fill=white!0!black] | & |[fill=white!0!black] |\\
    |[fill=white!0!black] | & |[fill=white!0!black] | & |[fill=white!0!black] | & |[fill=white!0!black] | & |[fill=white!0!black] | & |[fill=white!0!black] | & |[fill=white!0!black] | & |[fill=white!0!black] | & |[fill=white!0!black] | & |[fill=white!0!black] | & |[fill=white!0!black] | & |[fill=white!0!black] | & |[fill=white!0!black] | & |[fill=white!0!black] | & |[fill=white!0!black] | & |[fill=white!0!black] | & |[fill=white!0!black] | & |[fill=white!0!black] | & |[fill=white!0!black] | & |[fill=white!0!black] | & |[fill=white!0!black] | & |[fill=white!0!black] | & |[fill=white!0!black] | & |[fill=white!0!black] | & |[fill=white!0!black] | & |[fill=white!0!black] | & |[fill=white!0!black] | & |[fill=white!0!black] |\\
    |[fill=white!0!black] | & |[fill=white!0!black] | & |[fill=white!0!black] | & |[fill=white!0!black] | & |[fill=white!0!black] | & |[fill=white!0!black] | & |[fill=white!0!black] | & |[fill=white!0!black] | & |[fill=white!0!black] | & |[fill=white!0!black] | & |[fill=white!0!black] | & |[fill=white!0!black] | & |[fill=white!0!black] | & |[fill=white!0!black] | & |[fill=white!0!black] | & |[fill=white!0!black] | & |[fill=white!0!black] | & |[fill=white!0!black] | & |[fill=white!0!black] | & |[fill=white!0!black] | & |[fill=white!0!black] | & |[fill=white!53!black] | & |[fill=white!81!black] | & |[fill=white!0!black] | & |[fill=white!0!black] | & |[fill=white!0!black] | & |[fill=white!0!black] | & |[fill=white!0!black] |\\
    |[fill=white!0!black] | & |[fill=white!0!black] | & |[fill=white!0!black] | & |[fill=white!0!black] | & |[fill=white!0!black] | & |[fill=white!0!black] | & |[fill=white!0!black] | & |[fill=white!0!black] | & |[fill=white!0!black] | & |[fill=white!0!black] | & |[fill=white!0!black] | & |[fill=white!0!black] | & |[fill=white!0!black] | & |[fill=white!0!black] | & |[fill=white!0!black] | & |[fill=white!0!black] | & |[fill=white!0!black] | & |[fill=white!0!black] | & |[fill=white!0!black] | & |[fill=white!0!black] | & |[fill=white!27!black] | & |[fill=white!93!black] | & |[fill=white!81!black] | & |[fill=white!0!black] | & |[fill=white!0!black] | & |[fill=white!0!black] | & |[fill=white!0!black] | & |[fill=white!0!black] |\\
    |[fill=white!0!black] | & |[fill=white!0!black] | & |[fill=white!0!black] | & |[fill=white!0!black] | & |[fill=white!0!black] | & |[fill=white!0!black] | & |[fill=white!0!black] | & |[fill=white!0!black] | & |[fill=white!0!black] | & |[fill=white!0!black] | & |[fill=white!0!black] | & |[fill=white!0!black] | & |[fill=white!0!black] | & |[fill=white!0!black] | & |[fill=white!0!black] | & |[fill=white!0!black] | & |[fill=white!0!black] | & |[fill=white!0!black] | & |[fill=white!0!black] | & |[fill=white!1!black] | & |[fill=white!75!black] | & |[fill=white!100!black] | & |[fill=white!78!black] | & |[fill=white!0!black] | & |[fill=white!0!black] | & |[fill=white!0!black] | & |[fill=white!0!black] | & |[fill=white!0!black] |\\
    |[fill=white!0!black] | & |[fill=white!0!black] | & |[fill=white!0!black] | & |[fill=white!0!black] | & |[fill=white!0!black] | & |[fill=white!0!black] | & |[fill=white!0!black] | & |[fill=white!0!black] | & |[fill=white!0!black] | & |[fill=white!0!black] | & |[fill=white!0!black] | & |[fill=white!0!black] | & |[fill=white!0!black] | & |[fill=white!0!black] | & |[fill=white!0!black] | & |[fill=white!0!black] | & |[fill=white!0!black] | & |[fill=white!0!black] | & |[fill=white!0!black] | & |[fill=white!22!black] | & |[fill=white!99!black] | & |[fill=white!100!black] | & |[fill=white!19!black] | & |[fill=white!0!black] | & |[fill=white!0!black] | & |[fill=white!0!black] | & |[fill=white!0!black] | & |[fill=white!0!black] |\\
    |[fill=white!0!black] | & |[fill=white!0!black] | & |[fill=white!0!black] | & |[fill=white!0!black] | & |[fill=white!0!black] | & |[fill=white!0!black] | & |[fill=white!0!black] | & |[fill=white!0!black] | & |[fill=white!0!black] | & |[fill=white!0!black] | & |[fill=white!0!black] | & |[fill=white!0!black] | & |[fill=white!0!black] | & |[fill=white!0!black] | & |[fill=white!0!black] | & |[fill=white!0!black] | & |[fill=white!0!black] | & |[fill=white!0!black] | & |[fill=white!5!black] | & |[fill=white!73!black] | & |[fill=white!100!black] | & |[fill=white!82!black] | & |[fill=white!4!black] | & |[fill=white!0!black] | & |[fill=white!0!black] | & |[fill=white!0!black] | & |[fill=white!0!black] | & |[fill=white!0!black] |\\
    |[fill=white!0!black] | & |[fill=white!0!black] | & |[fill=white!0!black] | & |[fill=white!0!black] | & |[fill=white!0!black] | & |[fill=white!0!black] | & |[fill=white!0!black] | & |[fill=white!0!black] | & |[fill=white!0!black] | & |[fill=white!0!black] | & |[fill=white!0!black] | & |[fill=white!0!black] | & |[fill=white!0!black] | & |[fill=white!0!black] | & |[fill=white!0!black] | & |[fill=white!0!black] | & |[fill=white!0!black] | & |[fill=white!0!black] | & |[fill=white!38!black] | & |[fill=white!99!black] | & |[fill=white!86!black] | & |[fill=white!20!black] | & |[fill=white!0!black] | & |[fill=white!0!black] | & |[fill=white!0!black] | & |[fill=white!0!black] | & |[fill=white!0!black] | & |[fill=white!0!black] |\\
    |[fill=white!0!black] | & |[fill=white!0!black] | & |[fill=white!0!black] | & |[fill=white!0!black] | & |[fill=white!0!black] | & |[fill=white!0!black] | & |[fill=white!0!black] | & |[fill=white!0!black] | & |[fill=white!0!black] | & |[fill=white!0!black] | & |[fill=white!0!black] | & |[fill=white!0!black] | & |[fill=white!7!black] | & |[fill=white!22!black] | & |[fill=white!1!black] | & |[fill=white!0!black] | & |[fill=white!0!black] | & |[fill=white!26!black] | & |[fill=white!88!black] | & |[fill=white!99!black] | & |[fill=white!53!black] | & |[fill=white!0!black] | & |[fill=white!0!black] | & |[fill=white!0!black] | & |[fill=white!0!black] | & |[fill=white!0!black] | & |[fill=white!0!black] | & |[fill=white!0!black] |\\
    |[fill=white!0!black] | & |[fill=white!0!black] | & |[fill=white!0!black] | & |[fill=white!0!black] | & |[fill=white!0!black] | & |[fill=white!0!black] | & |[fill=white!0!black] | & |[fill=white!0!black] | & |[fill=white!0!black] | & |[fill=white!0!black] | & |[fill=white!0!black] | & |[fill=white!33!black] | & |[fill=white!92!black] | & |[fill=white!89!black] | & |[fill=white!6!black] | & |[fill=white!0!black] | & |[fill=white!0!black] | & |[fill=white!53!black] | & |[fill=white!99!black] | & |[fill=white!79!black] | & |[fill=white!7!black] | & |[fill=white!0!black] | & |[fill=white!0!black] | & |[fill=white!0!black] | & |[fill=white!0!black] | & |[fill=white!0!black] | & |[fill=white!0!black] | & |[fill=white!0!black] |\\
    |[fill=white!0!black] | & |[fill=white!0!black] | & |[fill=white!0!black] | & |[fill=white!0!black] | & |[fill=white!0!black] | & |[fill=white!0!black] | & |[fill=white!0!black] | & |[fill=white!0!black] | & |[fill=white!0!black] | & |[fill=white!0!black] | & |[fill=white!0!black] | & |[fill=white!36!black] | & |[fill=white!100!black] | & |[fill=white!59!black] | & |[fill=white!4!black] | & |[fill=white!0!black] | & |[fill=white!36!black] | & |[fill=white!100!black] | & |[fill=white!100!black] | & |[fill=white!44!black] | & |[fill=white!0!black] | & |[fill=white!0!black] | & |[fill=white!0!black] | & |[fill=white!0!black] | & |[fill=white!0!black] | & |[fill=white!0!black] | & |[fill=white!0!black] | & |[fill=white!0!black] |\\
    |[fill=white!0!black] | & |[fill=white!0!black] | & |[fill=white!0!black] | & |[fill=white!0!black] | & |[fill=white!0!black] | & |[fill=white!0!black] | & |[fill=white!0!black] | & |[fill=white!0!black] | & |[fill=white!0!black] | & |[fill=white!0!black] | & |[fill=white!0!black] | & |[fill=white!1!black] | & |[fill=white!35!black] | & |[fill=white!16!black] | & |[fill=white!0!black] | & |[fill=white!12!black] | & |[fill=white!84!black] | & |[fill=white!100!black] | & |[fill=white!63!black] | & |[fill=white!2!black] | & |[fill=white!0!black] | & |[fill=white!0!black] | & |[fill=white!0!black] | & |[fill=white!0!black] | & |[fill=white!0!black] | & |[fill=white!0!black] | & |[fill=white!0!black] | & |[fill=white!0!black] |\\
    |[fill=white!0!black] | & |[fill=white!0!black] | & |[fill=white!0!black] | & |[fill=white!0!black] | & |[fill=white!0!black] | & |[fill=white!0!black] | & |[fill=white!0!black] | & |[fill=white!0!black] | & |[fill=white!0!black] | & |[fill=white!29!black] | & |[fill=white!81!black] | & |[fill=white!65!black] | & |[fill=white!13!black] | & |[fill=white!0!black] | & |[fill=white!0!black] | & |[fill=white!62!black] | & |[fill=white!99!black] | & |[fill=white!77!black] | & |[fill=white!8!black] | & |[fill=white!0!black] | & |[fill=white!0!black] | & |[fill=white!0!black] | & |[fill=white!0!black] | & |[fill=white!0!black] | & |[fill=white!0!black] | & |[fill=white!0!black] | & |[fill=white!0!black] | & |[fill=white!0!black] |\\
    |[fill=white!0!black] | & |[fill=white!0!black] | & |[fill=white!0!black] | & |[fill=white!0!black] | & |[fill=white!0!black] | & |[fill=white!0!black] | & |[fill=white!0!black] | & |[fill=white!0!black] | & |[fill=white!27!black] | & |[fill=white!86!black] | & |[fill=white!99!black] | & |[fill=white!82!black] | & |[fill=white!17!black] | & |[fill=white!0!black] | & |[fill=white!20!black] | & |[fill=white!88!black] | & |[fill=white!86!black] | & |[fill=white!46!black] | & |[fill=white!0!black] | & |[fill=white!0!black] | & |[fill=white!0!black] | & |[fill=white!0!black] | & |[fill=white!0!black] | & |[fill=white!0!black] | & |[fill=white!0!black] | & |[fill=white!0!black] | & |[fill=white!0!black] | & |[fill=white!0!black] |\\
    |[fill=white!0!black] | & |[fill=white!0!black] | & |[fill=white!0!black] | & |[fill=white!0!black] | & |[fill=white!0!black] | & |[fill=white!0!black] | & |[fill=white!0!black] | & |[fill=white!16!black] | & |[fill=white!90!black] | & |[fill=white!100!black] | & |[fill=white!75!black] | & |[fill=white!13!black] | & |[fill=white!0!black] | & |[fill=white!0!black] | & |[fill=white!51!black] | & |[fill=white!100!black] | & |[fill=white!66!black] | & |[fill=white!0!black] | & |[fill=white!0!black] | & |[fill=white!0!black] | & |[fill=white!0!black] | & |[fill=white!0!black] | & |[fill=white!0!black] | & |[fill=white!0!black] | & |[fill=white!0!black] | & |[fill=white!0!black] | & |[fill=white!0!black] | & |[fill=white!0!black] |\\
    |[fill=white!0!black] | & |[fill=white!0!black] | & |[fill=white!0!black] | & |[fill=white!0!black] | & |[fill=white!0!black] | & |[fill=white!0!black] | & |[fill=white!10!black] | & |[fill=white!88!black] | & |[fill=white!99!black] | & |[fill=white!100!black] | & |[fill=white!25!black] | & |[fill=white!35!black] | & |[fill=white!54!black] | & |[fill=white!64!black] | & |[fill=white!98!black] | & |[fill=white!99!black] | & |[fill=white!66!black] | & |[fill=white!55!black] | & |[fill=white!64!black] | & |[fill=white!40!black] | & |[fill=white!0!black] | & |[fill=white!0!black] | & |[fill=white!0!black] | & |[fill=white!0!black] | & |[fill=white!0!black] | & |[fill=white!0!black] | & |[fill=white!0!black] | & |[fill=white!0!black] |\\
    |[fill=white!0!black] | & |[fill=white!0!black] | & |[fill=white!0!black] | & |[fill=white!0!black] | & |[fill=white!0!black] | & |[fill=white!0!black] | & |[fill=white!10!black] | & |[fill=white!91!black] | & |[fill=white!99!black] | & |[fill=white!100!black] | & |[fill=white!99!black] | & |[fill=white!99!black] | & |[fill=white!99!black] | & |[fill=white!100!black] | & |[fill=white!99!black] | & |[fill=white!92!black] | & |[fill=white!77!black] | & |[fill=white!54!black] | & |[fill=white!25!black] | & |[fill=white!0!black] | & |[fill=white!0!black] | & |[fill=white!0!black] | & |[fill=white!0!black] | & |[fill=white!0!black] | & |[fill=white!0!black] | & |[fill=white!0!black] | & |[fill=white!0!black] | & |[fill=white!0!black] |\\
    |[fill=white!0!black] | & |[fill=white!0!black] | & |[fill=white!0!black] | & |[fill=white!0!black] | & |[fill=white!0!black] | & |[fill=white!0!black] | & |[fill=white!0!black] | & |[fill=white!25!black] | & |[fill=white!37!black] | & |[fill=white!50!black] | & |[fill=white!63!black] | & |[fill=white!66!black] | & |[fill=white!99!black] | & |[fill=white!100!black] | & |[fill=white!59!black] | & |[fill=white!5!black] | & |[fill=white!0!black] | & |[fill=white!0!black] | & |[fill=white!0!black] | & |[fill=white!0!black] | & |[fill=white!0!black] | & |[fill=white!0!black] | & |[fill=white!0!black] | & |[fill=white!0!black] | & |[fill=white!0!black] | & |[fill=white!0!black] | & |[fill=white!0!black] | & |[fill=white!0!black] |\\
    |[fill=white!0!black] | & |[fill=white!0!black] | & |[fill=white!0!black] | & |[fill=white!0!black] | & |[fill=white!0!black] | & |[fill=white!0!black] | & |[fill=white!0!black] | & |[fill=white!0!black] | & |[fill=white!0!black] | & |[fill=white!0!black] | & |[fill=white!0!black] | & |[fill=white!55!black] | & |[fill=white!100!black] | & |[fill=white!63!black] | & |[fill=white!0!black] | & |[fill=white!0!black] | & |[fill=white!0!black] | & |[fill=white!0!black] | & |[fill=white!0!black] | & |[fill=white!0!black] | & |[fill=white!0!black] | & |[fill=white!0!black] | & |[fill=white!0!black] | & |[fill=white!0!black] | & |[fill=white!0!black] | & |[fill=white!0!black] | & |[fill=white!0!black] | & |[fill=white!0!black] |\\
    |[fill=white!0!black] | & |[fill=white!0!black] | & |[fill=white!0!black] | & |[fill=white!0!black] | & |[fill=white!0!black] | & |[fill=white!0!black] | & |[fill=white!0!black] | & |[fill=white!0!black] | & |[fill=white!0!black] | & |[fill=white!0!black] | & |[fill=white!17!black] | & |[fill=white!91!black] | & |[fill=white!86!black] | & |[fill=white!4!black] | & |[fill=white!0!black] | & |[fill=white!0!black] | & |[fill=white!0!black] | & |[fill=white!0!black] | & |[fill=white!0!black] | & |[fill=white!0!black] | & |[fill=white!0!black] | & |[fill=white!0!black] | & |[fill=white!0!black] | & |[fill=white!0!black] | & |[fill=white!0!black] | & |[fill=white!0!black] | & |[fill=white!0!black] | & |[fill=white!0!black] |\\
    |[fill=white!0!black] | & |[fill=white!0!black] | & |[fill=white!0!black] | & |[fill=white!0!black] | & |[fill=white!0!black] | & |[fill=white!0!black] | & |[fill=white!0!black] | & |[fill=white!0!black] | & |[fill=white!0!black] | & |[fill=white!0!black] | & |[fill=white!71!black] | & |[fill=white!99!black] | & |[fill=white!27!black] | & |[fill=white!0!black] | & |[fill=white!0!black] | & |[fill=white!0!black] | & |[fill=white!0!black] | & |[fill=white!0!black] | & |[fill=white!0!black] | & |[fill=white!0!black] | & |[fill=white!0!black] | & |[fill=white!0!black] | & |[fill=white!0!black] | & |[fill=white!0!black] | & |[fill=white!0!black] | & |[fill=white!0!black] | & |[fill=white!0!black] | & |[fill=white!0!black] |\\
    |[fill=white!0!black] | & |[fill=white!0!black] | & |[fill=white!0!black] | & |[fill=white!0!black] | & |[fill=white!0!black] | & |[fill=white!0!black] | & |[fill=white!0!black] | & |[fill=white!0!black] | & |[fill=white!0!black] | & |[fill=white!0!black] | & |[fill=white!64!black] | & |[fill=white!96!black] | & |[fill=white!31!black] | & |[fill=white!0!black] | & |[fill=white!0!black] | & |[fill=white!0!black] | & |[fill=white!0!black] | & |[fill=white!0!black] | & |[fill=white!0!black] | & |[fill=white!0!black] | & |[fill=white!0!black] | & |[fill=white!0!black] | & |[fill=white!0!black] | & |[fill=white!0!black] | & |[fill=white!0!black] | & |[fill=white!0!black] | & |[fill=white!0!black] | & |[fill=white!0!black] |\\
    |[fill=white!0!black] | & |[fill=white!0!black] | & |[fill=white!0!black] | & |[fill=white!0!black] | & |[fill=white!0!black] | & |[fill=white!0!black] | & |[fill=white!0!black] | & |[fill=white!0!black] | & |[fill=white!0!black] | & |[fill=white!0!black] | & |[fill=white!0!black] | & |[fill=white!0!black] | & |[fill=white!0!black] | & |[fill=white!0!black] | & |[fill=white!0!black] | & |[fill=white!0!black] | & |[fill=white!0!black] | & |[fill=white!0!black] | & |[fill=white!0!black] | & |[fill=white!0!black] | & |[fill=white!0!black] | & |[fill=white!0!black] | & |[fill=white!0!black] | & |[fill=white!0!black] | & |[fill=white!0!black] | & |[fill=white!0!black] | & |[fill=white!0!black] | & |[fill=white!0!black] |\\
    |[fill=white!0!black] | & |[fill=white!0!black] | & |[fill=white!0!black] | & |[fill=white!0!black] | & |[fill=white!0!black] | & |[fill=white!0!black] | & |[fill=white!0!black] | & |[fill=white!0!black] | & |[fill=white!0!black] | & |[fill=white!0!black] | & |[fill=white!0!black] | & |[fill=white!0!black] | & |[fill=white!0!black] | & |[fill=white!0!black] | & |[fill=white!0!black] | & |[fill=white!0!black] | & |[fill=white!0!black] | & |[fill=white!0!black] | & |[fill=white!0!black] | & |[fill=white!0!black] | & |[fill=white!0!black] | & |[fill=white!0!black] | & |[fill=white!0!black] | & |[fill=white!0!black] | & |[fill=white!0!black] | & |[fill=white!0!black] | & |[fill=white!0!black] | & |[fill=white!0!black] |\\
    |[fill=white!0!black] | & |[fill=white!0!black] | & |[fill=white!0!black] | & |[fill=white!0!black] | & |[fill=white!0!black] | & |[fill=white!0!black] | & |[fill=white!0!black] | & |[fill=white!0!black] | & |[fill=white!0!black] | & |[fill=white!0!black] | & |[fill=white!0!black] | & |[fill=white!0!black] | & |[fill=white!0!black] | & |[fill=white!0!black] | & |[fill=white!0!black] | & |[fill=white!0!black] | & |[fill=white!0!black] | & |[fill=white!0!black] | & |[fill=white!0!black] | & |[fill=white!0!black] | & |[fill=white!0!black] | & |[fill=white!0!black] | & |[fill=white!0!black] | & |[fill=white!0!black] | & |[fill=white!0!black] | & |[fill=white!0!black] | & |[fill=white!0!black] | & |[fill=white!0!black] |\\
    |[fill=white!0!black] | & |[fill=white!0!black] | & |[fill=white!0!black] | & |[fill=white!0!black] | & |[fill=white!0!black] | & |[fill=white!0!black] | & |[fill=white!0!black] | & |[fill=white!0!black] | & |[fill=white!0!black] | & |[fill=white!0!black] | & |[fill=white!0!black] | & |[fill=white!0!black] | & |[fill=white!0!black] | & |[fill=white!0!black] | & |[fill=white!0!black] | & |[fill=white!0!black] | & |[fill=white!0!black] | & |[fill=white!0!black] | & |[fill=white!0!black] | & |[fill=white!0!black] | & |[fill=white!0!black] | & |[fill=white!0!black] | & |[fill=white!0!black] | & |[fill=white!0!black] | & |[fill=white!0!black] | & |[fill=white!0!black] | & |[fill=white!0!black] | & |[fill=white!0!black] |\\
    };
\end{tikzpicture}
        }
        \caption{Das Input-Signal $G$.\\Schwarz $\hat=$ $0$, Weiß $\hat=$ $1$.}
    \end{subfigure}%
    \begin{subfigure}{0.5\textwidth}
        \resizebox{0.9285\textwidth}{0.9285\textwidth}{
        
\tikzset{ 
table/.style={
    matrix of nodes,
    nodes={rectangle,minimum size=0.05cm,align=center},
    nodes in empty cells
    }
}
\begin{tikzpicture}
    \matrix (mat) [table]
    {
    |[fill=white!0!black] | & |[fill=white!0!black] | & |[fill=white!0!black] | & |[fill=white!0!black] | & |[fill=white!0!black] | & |[fill=white!0!black] | & |[fill=white!0!black] | & |[fill=white!0!black] | & |[fill=white!0!black] | & |[fill=white!0!black] | & |[fill=white!0!black] | & |[fill=white!0!black] | & |[fill=white!0!black] | & |[fill=white!0!black] | & |[fill=white!0!black] | & |[fill=white!0!black] | & |[fill=white!0!black] | & |[fill=white!0!black] | & |[fill=white!0!black] | & |[fill=white!0!black] | & |[fill=white!0!black] | & |[fill=white!0!black] | & |[fill=white!0!black] | & |[fill=white!0!black] | & |[fill=white!0!black] | & |[fill=white!0!black] | & |[fill=white!0!black] | & |[fill=white!0!black] |\\
    |[fill=white!0!black] | & |[fill=white!0!black] | & |[fill=white!0!black] | & |[fill=white!0!black] | & |[fill=white!0!black] | & |[fill=white!0!black] | & |[fill=white!0!black] | & |[fill=white!0!black] | & |[fill=white!0!black] | & |[fill=white!0!black] | & |[fill=white!0!black] | & |[fill=white!0!black] | & |[fill=white!0!black] | & |[fill=white!0!black] | & |[fill=white!0!black] | & |[fill=white!0!black] | & |[fill=white!0!black] | & |[fill=white!0!black] | & |[fill=white!0!black] | & |[fill=white!0!black] | & |[fill=white!0!black] | & |[fill=white!0!black] | & |[fill=white!0!black] | & |[fill=white!0!black] | & |[fill=white!0!black] | & |[fill=white!0!black] | & |[fill=white!0!black] | & |[fill=white!0!black] |\\
    |[fill=white!0!black] | & |[fill=white!0!black] | & |[fill=white!0!black] | & |[fill=white!0!black] | & |[fill=white!0!black] | & |[fill=white!0!black] | & |[fill=white!0!black] | & |[fill=white!0!black] | & |[fill=white!0!black] | & |[fill=white!0!black] | & |[fill=white!0!black] | & |[fill=white!0!black] | & |[fill=white!0!black] | & |[fill=white!0!black] | & |[fill=white!0!black] | & |[fill=white!0!black] | & |[fill=white!0!black] | & |[fill=white!0!black] | & |[fill=white!0!black] | & |[fill=white!0!black] | & |[fill=white!0!black] | & |[fill=white!0!black] | & |[fill=white!0!black] | & |[fill=white!0!black] | & |[fill=white!0!black] | & |[fill=white!0!black] | & |[fill=white!0!black] | & |[fill=white!0!black] |\\
    |[fill=white!0!black] | & |[fill=white!0!black] | & |[fill=white!0!black] | & |[fill=white!0!black] | & |[fill=white!0!black] | & |[fill=white!0!black] | & |[fill=white!0!black] | & |[fill=white!0!black] | & |[fill=white!0!black] | & |[fill=white!0!black] | & |[fill=white!0!black] | & |[fill=white!0!black] | & |[fill=white!0!black] | & |[fill=white!0!black] | & |[fill=white!0!black] | & |[fill=white!0!black] | & |[fill=white!0!black] | & |[fill=white!0!black] | & |[fill=white!0!black] | & |[fill=white!0!black] | & |[fill=white!0!black] | & |[fill=white!0!black] | & |[fill=white!0!black] | & |[fill=white!0!black] | & |[fill=white!0!black] | & |[fill=white!0!black] | & |[fill=white!0!black] | & |[fill=white!0!black] |\\
    |[fill=white!0!black] | & |[fill=white!0!black] | & |[fill=white!0!black] | & |[fill=white!0!black] | & |[fill=white!0!black] | & |[fill=white!0!black] | & |[fill=white!0!black] | & |[fill=white!0!black] | & |[fill=white!0!black] | & |[fill=white!0!black] | & |[fill=white!0!black] | & |[fill=white!0!black] | & |[fill=white!0!black] | & |[fill=white!0!black] | & |[fill=white!0!black] | & |[fill=white!0!black] | & |[fill=white!0!black] | & |[fill=white!0!black] | & |[fill=white!0!black] | & |[fill=white!0!black] | & |[fill=white!0!black] | & |[fill=white!0!black] | & |[fill=white!0!black] | & |[fill=white!0!black] | & |[fill=white!0!black] | & |[fill=white!0!black] | & |[fill=white!0!black] | & |[fill=white!0!black] |\\
    |[fill=white!0!black] | & |[fill=white!0!black] | & |[fill=white!0!black] | & |[fill=white!0!black] | & |[fill=white!0!black] | & |[fill=white!0!black] | & |[fill=white!0!black] | & |[fill=white!0!black] | & |[fill=white!0!black] | & |[fill=white!0!black] | & |[fill=white!0!black] | & |[fill=white!0!black] | & |[fill=white!0!black] | & |[fill=white!0!black] | & |[fill=white!0!black] | & |[fill=white!0!black] | & |[fill=white!0!black] | & |[fill=white!0!black] | & |[fill=white!0!black] | & |[fill=white!0!black] | & |[fill=white!0!black] | & |[fill=white!0!black] | & |[fill=white!0!black] | & |[fill=white!3!black] | & |[fill=white!0!black] | & |[fill=white!0!black] | & |[fill=white!0!black] | & |[fill=white!0!black] |\\
    |[fill=white!0!black] | & |[fill=white!0!black] | & |[fill=white!0!black] | & |[fill=white!0!black] | & |[fill=white!0!black] | & |[fill=white!0!black] | & |[fill=white!0!black] | & |[fill=white!0!black] | & |[fill=white!0!black] | & |[fill=white!0!black] | & |[fill=white!0!black] | & |[fill=white!0!black] | & |[fill=white!0!black] | & |[fill=white!0!black] | & |[fill=white!0!black] | & |[fill=white!0!black] | & |[fill=white!0!black] | & |[fill=white!0!black] | & |[fill=white!0!black] | & |[fill=white!0!black] | & |[fill=white!0!black] | & |[fill=white!0!black] | & |[fill=white!47!black] | & |[fill=white!53!black] | & |[fill=white!0!black] | & |[fill=white!0!black] | & |[fill=white!0!black] | & |[fill=white!0!black] |\\
    |[fill=white!0!black] | & |[fill=white!0!black] | & |[fill=white!0!black] | & |[fill=white!0!black] | & |[fill=white!0!black] | & |[fill=white!0!black] | & |[fill=white!0!black] | & |[fill=white!0!black] | & |[fill=white!0!black] | & |[fill=white!0!black] | & |[fill=white!0!black] | & |[fill=white!0!black] | & |[fill=white!0!black] | & |[fill=white!0!black] | & |[fill=white!0!black] | & |[fill=white!0!black] | & |[fill=white!0!black] | & |[fill=white!0!black] | & |[fill=white!0!black] | & |[fill=white!0!black] | & |[fill=white!0!black] | & |[fill=white!57!black] | & |[fill=white!78!black] | & |[fill=white!63!black] | & |[fill=white!0!black] | & |[fill=white!0!black] | & |[fill=white!0!black] | & |[fill=white!0!black] |\\
    |[fill=white!0!black] | & |[fill=white!0!black] | & |[fill=white!0!black] | & |[fill=white!0!black] | & |[fill=white!0!black] | & |[fill=white!0!black] | & |[fill=white!0!black] | & |[fill=white!0!black] | & |[fill=white!0!black] | & |[fill=white!0!black] | & |[fill=white!0!black] | & |[fill=white!0!black] | & |[fill=white!0!black] | & |[fill=white!0!black] | & |[fill=white!0!black] | & |[fill=white!0!black] | & |[fill=white!0!black] | & |[fill=white!0!black] | & |[fill=white!0!black] | & |[fill=white!0!black] | & |[fill=white!14!black] | & |[fill=white!95!black] | & |[fill=white!84!black] | & |[fill=white!16!black] | & |[fill=white!0!black] | & |[fill=white!0!black] | & |[fill=white!0!black] | & |[fill=white!0!black] |\\
    |[fill=white!0!black] | & |[fill=white!0!black] | & |[fill=white!0!black] | & |[fill=white!0!black] | & |[fill=white!0!black] | & |[fill=white!0!black] | & |[fill=white!0!black] | & |[fill=white!0!black] | & |[fill=white!0!black] | & |[fill=white!0!black] | & |[fill=white!0!black] | & |[fill=white!0!black] | & |[fill=white!0!black] | & |[fill=white!0!black] | & |[fill=white!0!black] | & |[fill=white!0!black] | & |[fill=white!0!black] | & |[fill=white!0!black] | & |[fill=white!0!black] | & |[fill=white!0!black] | & |[fill=white!88!black] | & |[fill=white!113!black] | & |[fill=white!73!black] | & |[fill=white!3!black] | & |[fill=white!0!black] | & |[fill=white!0!black] | & |[fill=white!0!black] | & |[fill=white!0!black] |\\
    |[fill=white!0!black] | & |[fill=white!0!black] | & |[fill=white!0!black] | & |[fill=white!0!black] | & |[fill=white!0!black] | & |[fill=white!0!black] | & |[fill=white!0!black] | & |[fill=white!0!black] | & |[fill=white!0!black] | & |[fill=white!0!black] | & |[fill=white!0!black] | & |[fill=white!0!black] | & |[fill=white!0!black] | & |[fill=white!0!black] | & |[fill=white!0!black] | & |[fill=white!0!black] | & |[fill=white!0!black] | & |[fill=white!0!black] | & |[fill=white!0!black] | & |[fill=white!32!black] | & |[fill=white!100!black] | & |[fill=white!83!black] | & |[fill=white!17!black] | & |[fill=white!0!black] | & |[fill=white!0!black] | & |[fill=white!0!black] | & |[fill=white!0!black] | & |[fill=white!0!black] |\\
    |[fill=white!0!black] | & |[fill=white!0!black] | & |[fill=white!0!black] | & |[fill=white!0!black] | & |[fill=white!0!black] | & |[fill=white!0!black] | & |[fill=white!0!black] | & |[fill=white!0!black] | & |[fill=white!0!black] | & |[fill=white!0!black] | & |[fill=white!0!black] | & |[fill=white!0!black] | & |[fill=white!0!black] | & |[fill=white!0!black] | & |[fill=white!0!black] | & |[fill=white!0!black] | & |[fill=white!0!black] | & |[fill=white!0!black] | & |[fill=white!0!black] | & |[fill=white!82!black] | & |[fill=white!92!black] | & |[fill=white!45!black] | & |[fill=white!0!black] | & |[fill=white!0!black] | & |[fill=white!0!black] | & |[fill=white!0!black] | & |[fill=white!0!black] | & |[fill=white!0!black] |\\
    |[fill=white!0!black] | & |[fill=white!0!black] | & |[fill=white!0!black] | & |[fill=white!0!black] | & |[fill=white!0!black] | & |[fill=white!0!black] | & |[fill=white!0!black] | & |[fill=white!0!black] | & |[fill=white!0!black] | & |[fill=white!0!black] | & |[fill=white!27!black] | & |[fill=white!75!black] | & |[fill=white!137!black] | & |[fill=white!115!black] | & |[fill=white!57!black] | & |[fill=white!0!black] | & |[fill=white!0!black] | & |[fill=white!0!black] | & |[fill=white!58!black] | & |[fill=white!103!black] | & |[fill=white!72!black] | & |[fill=white!6!black] | & |[fill=white!0!black] | & |[fill=white!0!black] | & |[fill=white!0!black] | & |[fill=white!0!black] | & |[fill=white!0!black] | & |[fill=white!0!black] |\\
    |[fill=white!0!black] | & |[fill=white!0!black] | & |[fill=white!0!black] | & |[fill=white!0!black] | & |[fill=white!0!black] | & |[fill=white!0!black] | & |[fill=white!0!black] | & |[fill=white!0!black] | & |[fill=white!0!black] | & |[fill=white!0!black] | & |[fill=white!0!black] | & |[fill=white!0!black] | & |[fill=white!100!black] | & |[fill=white!127!black] | & |[fill=white!1!black] | & |[fill=white!0!black] | & |[fill=white!0!black] | & |[fill=white!43!black] | & |[fill=white!134!black] | & |[fill=white!115!black] | & |[fill=white!38!black] | & |[fill=white!0!black] | & |[fill=white!0!black] | & |[fill=white!0!black] | & |[fill=white!0!black] | & |[fill=white!0!black] | & |[fill=white!0!black] | & |[fill=white!0!black] |\\
    |[fill=white!0!black] | & |[fill=white!0!black] | & |[fill=white!0!black] | & |[fill=white!0!black] | & |[fill=white!0!black] | & |[fill=white!0!black] | & |[fill=white!0!black] | & |[fill=white!0!black] | & |[fill=white!0!black] | & |[fill=white!0!black] | & |[fill=white!0!black] | & |[fill=white!0!black] | & |[fill=white!0!black] | & |[fill=white!12!black] | & |[fill=white!0!black] | & |[fill=white!0!black] | & |[fill=white!0!black] | & |[fill=white!97!black] | & |[fill=white!100!black] | & |[fill=white!55!black] | & |[fill=white!1!black] | & |[fill=white!0!black] | & |[fill=white!0!black] | & |[fill=white!0!black] | & |[fill=white!0!black] | & |[fill=white!0!black] | & |[fill=white!0!black] | & |[fill=white!0!black] |\\
    |[fill=white!0!black] | & |[fill=white!0!black] | & |[fill=white!0!black] | & |[fill=white!0!black] | & |[fill=white!0!black] | & |[fill=white!0!black] | & |[fill=white!0!black] | & |[fill=white!0!black] | & |[fill=white!0!black] | & |[fill=white!0!black] | & |[fill=white!0!black] | & |[fill=white!60!black] | & |[fill=white!54!black] | & |[fill=white!0!black] | & |[fill=white!0!black] | & |[fill=white!0!black] | & |[fill=white!61!black] | & |[fill=white!101!black] | & |[fill=white!73!black] | & |[fill=white!7!black] | & |[fill=white!0!black] | & |[fill=white!0!black] | & |[fill=white!0!black] | & |[fill=white!0!black] | & |[fill=white!0!black] | & |[fill=white!0!black] | & |[fill=white!0!black] | & |[fill=white!0!black] |\\
    |[fill=white!0!black] | & |[fill=white!0!black] | & |[fill=white!0!black] | & |[fill=white!0!black] | & |[fill=white!0!black] | & |[fill=white!0!black] | & |[fill=white!0!black] | & |[fill=white!0!black] | & |[fill=white!0!black] | & |[fill=white!0!black] | & |[fill=white!92!black] | & |[fill=white!72!black] | & |[fill=white!0!black] | & |[fill=white!0!black] | & |[fill=white!0!black] | & |[fill=white!0!black] | & |[fill=white!0!black] | & |[fill=white!0!black] | & |[fill=white!0!black] | & |[fill=white!0!black] | & |[fill=white!0!black] | & |[fill=white!0!black] | & |[fill=white!0!black] | & |[fill=white!0!black] | & |[fill=white!0!black] | & |[fill=white!0!black] | & |[fill=white!0!black] | & |[fill=white!0!black] |\\
    |[fill=white!0!black] | & |[fill=white!0!black] | & |[fill=white!0!black] | & |[fill=white!0!black] | & |[fill=white!0!black] | & |[fill=white!0!black] | & |[fill=white!0!black] | & |[fill=white!0!black] | & |[fill=white!0!black] | & |[fill=white!0!black] | & |[fill=white!0!black] | & |[fill=white!0!black] | & |[fill=white!0!black] | & |[fill=white!0!black] | & |[fill=white!0!black] | & |[fill=white!0!black] | & |[fill=white!0!black] | & |[fill=white!0!black] | & |[fill=white!0!black] | & |[fill=white!0!black] | & |[fill=white!0!black] | & |[fill=white!0!black] | & |[fill=white!0!black] | & |[fill=white!0!black] | & |[fill=white!0!black] | & |[fill=white!0!black] | & |[fill=white!0!black] | & |[fill=white!0!black] |\\
    |[fill=white!0!black] | & |[fill=white!0!black] | & |[fill=white!0!black] | & |[fill=white!0!black] | & |[fill=white!0!black] | & |[fill=white!9!black] | & |[fill=white!62!black] | & |[fill=white!115!black] | & |[fill=white!148!black] | & |[fill=white!63!black] | & |[fill=white!0!black] | & |[fill=white!0!black] | & |[fill=white!0!black] | & |[fill=white!0!black] | & |[fill=white!83!black] | & |[fill=white!170!black] | & |[fill=white!183!black] | & |[fill=white!157!black] | & |[fill=white!134!black] | & |[fill=white!88!black] | & |[fill=white!34!black] | & |[fill=white!0!black] | & |[fill=white!0!black] | & |[fill=white!0!black] | & |[fill=white!0!black] | & |[fill=white!0!black] | & |[fill=white!0!black] | & |[fill=white!0!black] |\\
    |[fill=white!0!black] | & |[fill=white!0!black] | & |[fill=white!0!black] | & |[fill=white!0!black] | & |[fill=white!0!black] | & |[fill=white!9!black] | & |[fill=white!86!black] | & |[fill=white!170!black] | & |[fill=white!246!black] | & |[fill=white!253!black] | & |[fill=white!206!black] | & |[fill=white!121!black] | & |[fill=white!68!black] | & |[fill=white!115!black] | & |[fill=white!194!black] | & |[fill=white!228!black] | & |[fill=white!189!black] | & |[fill=white!132!black] | & |[fill=white!67!black] | & |[fill=white!21!black] | & |[fill=white!0!black] | & |[fill=white!0!black] | & |[fill=white!0!black] | & |[fill=white!0!black] | & |[fill=white!0!black] | & |[fill=white!0!black] | & |[fill=white!0!black] | & |[fill=white!0!black] |\\
    |[fill=white!0!black] | & |[fill=white!0!black] | & |[fill=white!0!black] | & |[fill=white!0!black] | & |[fill=white!0!black] | & |[fill=white!0!black] | & |[fill=white!22!black] | & |[fill=white!53!black] | & |[fill=white!96!black] | & |[fill=white!113!black] | & |[fill=white!61!black] | & |[fill=white!29!black] | & |[fill=white!72!black] | & |[fill=white!143!black] | & |[fill=white!136!black] | & |[fill=white!55!black] | & |[fill=white!4!black] | & |[fill=white!0!black] | & |[fill=white!0!black] | & |[fill=white!0!black] | & |[fill=white!0!black] | & |[fill=white!0!black] | & |[fill=white!0!black] | & |[fill=white!0!black] | & |[fill=white!0!black] | & |[fill=white!0!black] | & |[fill=white!0!black] | & |[fill=white!0!black] |\\
    |[fill=white!0!black] | & |[fill=white!0!black] | & |[fill=white!0!black] | & |[fill=white!0!black] | & |[fill=white!0!black] | & |[fill=white!0!black] | & |[fill=white!0!black] | & |[fill=white!0!black] | & |[fill=white!0!black] | & |[fill=white!0!black] | & |[fill=white!0!black] | & |[fill=white!0!black] | & |[fill=white!78!black] | & |[fill=white!115!black] | & |[fill=white!53!black] | & |[fill=white!0!black] | & |[fill=white!0!black] | & |[fill=white!0!black] | & |[fill=white!0!black] | & |[fill=white!0!black] | & |[fill=white!0!black] | & |[fill=white!0!black] | & |[fill=white!0!black] | & |[fill=white!0!black] | & |[fill=white!0!black] | & |[fill=white!0!black] | & |[fill=white!0!black] | & |[fill=white!0!black] |\\
    |[fill=white!0!black] | & |[fill=white!0!black] | & |[fill=white!0!black] | & |[fill=white!0!black] | & |[fill=white!0!black] | & |[fill=white!0!black] | & |[fill=white!0!black] | & |[fill=white!0!black] | & |[fill=white!0!black] | & |[fill=white!0!black] | & |[fill=white!0!black] | & |[fill=white!3!black] | & |[fill=white!46!black] | & |[fill=white!50!black] | & |[fill=white!3!black] | & |[fill=white!0!black] | & |[fill=white!0!black] | & |[fill=white!0!black] | & |[fill=white!0!black] | & |[fill=white!0!black] | & |[fill=white!0!black] | & |[fill=white!0!black] | & |[fill=white!0!black] | & |[fill=white!0!black] | & |[fill=white!0!black] | & |[fill=white!0!black] | & |[fill=white!0!black] | & |[fill=white!0!black] |\\
    |[fill=white!0!black] | & |[fill=white!0!black] | & |[fill=white!0!black] | & |[fill=white!0!black] | & |[fill=white!0!black] | & |[fill=white!0!black] | & |[fill=white!0!black] | & |[fill=white!0!black] | & |[fill=white!0!black] | & |[fill=white!60!black] | & |[fill=white!144!black] | & |[fill=white!167!black] | & |[fill=white!107!black] | & |[fill=white!23!black] | & |[fill=white!0!black] | & |[fill=white!0!black] | & |[fill=white!0!black] | & |[fill=white!0!black] | & |[fill=white!0!black] | & |[fill=white!0!black] | & |[fill=white!0!black] | & |[fill=white!0!black] | & |[fill=white!0!black] | & |[fill=white!0!black] | & |[fill=white!0!black] | & |[fill=white!0!black] | & |[fill=white!0!black] | & |[fill=white!0!black] |\\
    |[fill=white!0!black] | & |[fill=white!0!black] | & |[fill=white!0!black] | & |[fill=white!0!black] | & |[fill=white!0!black] | & |[fill=white!0!black] | & |[fill=white!0!black] | & |[fill=white!0!black] | & |[fill=white!0!black] | & |[fill=white!54!black] | & |[fill=white!136!black] | & |[fill=white!162!black] | & |[fill=white!107!black] | & |[fill=white!26!black] | & |[fill=white!0!black] | & |[fill=white!0!black] | & |[fill=white!0!black] | & |[fill=white!0!black] | & |[fill=white!0!black] | & |[fill=white!0!black] | & |[fill=white!0!black] | & |[fill=white!0!black] | & |[fill=white!0!black] | & |[fill=white!0!black] | & |[fill=white!0!black] | & |[fill=white!0!black] | & |[fill=white!0!black] | & |[fill=white!0!black] |\\
    |[fill=white!0!black] | & |[fill=white!0!black] | & |[fill=white!0!black] | & |[fill=white!0!black] | & |[fill=white!0!black] | & |[fill=white!0!black] | & |[fill=white!0!black] | & |[fill=white!0!black] | & |[fill=white!0!black] | & |[fill=white!0!black] | & |[fill=white!0!black] | & |[fill=white!0!black] | & |[fill=white!0!black] | & |[fill=white!0!black] | & |[fill=white!0!black] | & |[fill=white!0!black] | & |[fill=white!0!black] | & |[fill=white!0!black] | & |[fill=white!0!black] | & |[fill=white!0!black] | & |[fill=white!0!black] | & |[fill=white!0!black] | & |[fill=white!0!black] | & |[fill=white!0!black] | & |[fill=white!0!black] | & |[fill=white!0!black] | & |[fill=white!0!black] | & |[fill=white!0!black] |\\
    |[fill=white!0!black] | & |[fill=white!0!black] | & |[fill=white!0!black] | & |[fill=white!0!black] | & |[fill=white!0!black] | & |[fill=white!0!black] | & |[fill=white!0!black] | & |[fill=white!0!black] | & |[fill=white!0!black] | & |[fill=white!0!black] | & |[fill=white!0!black] | & |[fill=white!0!black] | & |[fill=white!0!black] | & |[fill=white!0!black] | & |[fill=white!0!black] | & |[fill=white!0!black] | & |[fill=white!0!black] | & |[fill=white!0!black] | & |[fill=white!0!black] | & |[fill=white!0!black] | & |[fill=white!0!black] | & |[fill=white!0!black] | & |[fill=white!0!black] | & |[fill=white!0!black] | & |[fill=white!0!black] | & |[fill=white!0!black] | & |[fill=white!0!black] | & |[fill=white!0!black] |\\
    |[fill=white!0!black] | & |[fill=white!0!black] | & |[fill=white!0!black] | & |[fill=white!0!black] | & |[fill=white!0!black] | & |[fill=white!0!black] | & |[fill=white!0!black] | & |[fill=white!0!black] | & |[fill=white!0!black] | & |[fill=white!0!black] | & |[fill=white!0!black] | & |[fill=white!0!black] | & |[fill=white!0!black] | & |[fill=white!0!black] | & |[fill=white!0!black] | & |[fill=white!0!black] | & |[fill=white!0!black] | & |[fill=white!0!black] | & |[fill=white!0!black] | & |[fill=white!0!black] | & |[fill=white!0!black] | & |[fill=white!0!black] | & |[fill=white!0!black] | & |[fill=white!0!black] | & |[fill=white!0!black] | & |[fill=white!0!black] | & |[fill=white!0!black] | & |[fill=white!0!black] |\\
    };
\end{tikzpicture}

        }
    \caption{Die Kreuzkorrelation $F\star G$.\\Schwarz $\hat=$ $-3{,}84$, Weiß $\hat=$ $3{,}98$.}
    \end{subfigure}
    \caption{Die Kreuzkorrelation einer handgeschriebenen Ziffer mit dem Sobel-Filter $F$ zur Erkennung von horizontalen Kanten.}
    \label{fig:sobel-on-mnist}
\end{figure}

Bei der Berechnung der Kreuzkorrelation $F\star G$ mit einem Eingangssignal $G$, hilft die Vorstellung, dass der Kern $F$ schrittweise über den Input geschoben wird und in jedem Schritt ein Output-Pixel als die durch den Kern gewichtete Summe der Eingangspixel berechnet wird.
Dabei erhält man
\[
(F\star G)_{i,j} = \begin{matrix*}[l]
    \hphantom{+}1\cdot G_{i,j}  & + 2\cdot G_{i, j+1} & + 1\cdot G_{i, j+2} \\
    + 0\cdot G_{i+1, j} &+ 0\cdot G_{i+1, j+1} &+ 0\cdot G_{i+1, j+2} \\
    +1\cdot G_{i+1, j} &+ 2\cdot G_{i+1, j+1} &+ 1\cdot G_{i+2, j+2}.
    \end{matrix*}
\]

In Abbildung~\ref{fig:sobel-on-mnist} sieht man die Kreuzkorrelation des Sobel-Filters am Beispiel einer handschriftlichen Ziffer aus der MNIST-Datenbank aus~\cite{lecun2010mnist}.
Das Ausgangsbild $G$ hat $28\times 28$ Pixel.
Wir betrachten zumeist nur solche Pixel von $F\star G$, in denen der gesamte Filter im \glqq relevanten Teil\grqq\ des Inputs, also innerhalb dieser $28\times 28$ Pixel ist.
Daher hat das Output-Bild in beiden Längen $2$ Pixel weniger als $G$.
Diese Technik nennt man Kreuzkorrelation \emph{ohne Padding}.
\todo{In Abschnitt bla werden weitere Varianten der Kreuzkorrelation beleuchtet.}

\section{Faltungsnetze}





\begin{definition}[The Convolution Operation]
    
\end{definition}

\clearpage          % neue Seite für Literaturverzeichnis
\thispagestyle{empty}
\bibliography{literature}

\end{document}